\section{Conclusion}\label{sec:conclusion}
In this paper, we studied the problem of continuous-time relationship prediction in both dynamic and heterogeneous information networks. To effectively tackle this problem, we first introduced a novel feature extraction framework based on meta-path modeling and recurrent neural network autoencoders to systematically extract features that take both the temporal dynamics and heterogeneous characteristics of the network into account for solving the continuous-time relationship problem. We then proposed a supervised non-parametric model, called \npglm, which exploits the extracted features to predict the relationship building time in information networks. The strength of our model is that it does not impose any significant assumptions on the underlying distribution of the relationship building time given its features, but tries to infer it from the data via a non-parametric approach. Extensive experiments conducted on a synthetic dataset and real-world datasets from DBLP, Delicious, and MovieLens demonstrated the correctness of our method and its effectiveness in predicting the relationship building time.

{For future work, we would like to design a unified architecture to combine feature extraction step with the learning algorithm in an integrated deep learning framework. Moreover, although the propsed method is able to scale to large information networks with thousands of nodes, it is not currently extensible to web-scale information networks where the number of nodes is in the scale of hundreds of millions. Learning temporal non-parametric models within an extremely huge dataset is a challenging problem and is an interesting and important future work. As calculating meta-path-based features are the primary computational bottleneck of our method, to make the learning process scalable, we set to investigate node embedding and approximation techniques.}
