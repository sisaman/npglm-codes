\section{Conclusion}\label{sec:conclusion}
In this paper, we studied the problem of continuous-time relationship prediction in both dynamic and heterogeneous information networks. To effectively tackle this problem, we first introduced a novel feature extraction framework based on meta-path modeling and recurrent neural network autoencoders to systematically extract features that take both the temporal dynamics and heterogeneous characteristics of the network into account for solving the continuous-time relationship problem. We then proposed a supervised non-parametric model, called \npglm, which exploits the extracted features to predict the relationship building time in information networks. The strength of our model is that it does not impose any significant assumptions on the underlying distribution of the relationship building time given its features, but tries to infer it from the data via a non-parametric approach. Extensive experiments conducted on a synthetic dataset and real-world datasets from DBLP, Delicious, and MovieLens demonstrated the correctness of our method and its effectiveness in predicting the relationship building time.
