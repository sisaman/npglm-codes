\begin{figure*}
	\centering
	\scriptsize
	\tikzstyle{block} = [ellipse,draw=black]
	\tikzstyle{arrow} = [thick,->,>=stealth]
	\tikzstyle{label} = [fill=white,inner sep=0,xshift=0.1cm,yshift=.03cm]
	\tikzstyle{self} = [out=-110,in=-70,loop,shorten >=1pt]
	\subfloat[DBLP\label{fig:schema:dblp}]{
		\begin{tikzpicture}
		\node[block] (P) at (0,0) {$\underline{P}aper$};
		\node[block] (V) at (0,1.25) {$\underline{V}enue$};
		\node[block] (T) at (-2.4,0) {$\underline{T}erm$};
		\node[block] (A) at (2.4,0) {$\underline{A}uthor$};
		\node(hidden) [draw=none] at (0,-1.5){};
		
		\draw [arrow] (A) -- node[label] {write}   (P);
		\draw [arrow] (V) -- node[label] {publish} (P);           
		\draw [arrow] (P) -- node[label] {mention} (T);
		\draw [arrow] (P) to [self] node[label,yshift=-.2cm] {cite} (P);
		
		\end{tikzpicture}
	}
	\hfil
	\subfloat[Delicious\label{fig:schema:delicious}]{
		\begin{tikzpicture}
		
		\node(B) [block] at (0,0) {$\underline{B}ookmark$};
		\node(T) [block] at (-2.4,0) {$\underline{T}ag$};
		\node(U) [block] at (2.4,0) {$\underline{U}ser$};
		\node(hidden) [draw=none] at (0,-1.5){};
		
		\draw [arrow] (U) -- node[label]{post} (B);
		\draw [arrow] (B) -- node[label]{has-tag} (T);
		\draw [arrow] (U) to [self] node[label,yshift=-.2cm]{contact} (U);
		
		\end{tikzpicture}
		%		\vspace{1cm}
	}
	\hfil
	\subfloat[MovieLens\label{fig:schema:movielens}]{
		\begin{tikzpicture}
		
		\node[block] (M) at (0,0) {$\underline{M}ovie$};
		\node[block] (U) at (2.2,0) {$\underline{U}ser$};
		\node[block] (C) at (-1.25,1.25) {$\underline{C}ountry$};
		\node[block] (T) at (-2.2,0) {$\underline{T}ag$};
		\node[block] (G) at (1.25,1.25) {$\underline{G}enre$};
		\node[block] (A) at (1.25,-1.25) {$\underline{A}ctor$};
		\node[block] (D) at (-1.25,-1.25) {$\underline{D}irector$};
		\node(hidden) [draw=none] at (0,-1.5){};
		
		\draw [arrow] (U) -- node[label] {rate} (M);
		\draw [arrow] (M) -- node[label] {has-tag} (T);
		\draw [arrow] (M) -- node[label] {has-genre} (G);
		\draw [arrow] (A) -- node[label] {play-in} (M);
		\draw [arrow] (D) -- node[label] {direct} (M);
		\draw [arrow] (M) -- node[label] {produced-in} (C);
		
		\end{tikzpicture}
	}
	\caption{Schema of three different heterogeneous networks. Underlined characters are used as abbreviations for corresponding node types. }
	\label{fig:schema}
\end{figure*}

\section{Problem Formulation}\label{sec:problem}
In this section, we introduce some important concepts and definitions used throughout the paper and formally define the problem of continuous-time relationship prediction.

\subsection{Heterogeneous Information Networks}

An information network is \emph{heterogeneous} if it contains multiple kinds of nodes and links. Formally, it is defined as a directed graph $G=(V,E)$ where $V = \bigcup_i V_i$ is the set of nodes comprising the union of all the node sets $V_i$ of type $i$. Similarly, $E=\bigcup_j E_j$ is the set of links constituted by the union of all the link sets $E_j$ of type $j$. Now we bring the definition of the \emph{network schema} \cite{sun2011pathsim} which is used to describe a heterogeneous information network at a meta-level:

\begin{definition}{(Network Schema)}
	The schema of a heterogeneous network $G$ is a graph $\mc{S}_G=(\mc{V}, \mc{E})$ where $\mc{V}$ is the set of different node types and $\mc{E}$ is the set of different link types in $G$.
\end{definition}

In this paper, we focus on three different heterogeneous and dynamic networks: (1) DBLP bibliographic network\footnote{http://dblp.uni-trier.de/}; (2) Delicious bookmarking network\footnote{http://delicious.com/}; and (3) MovieLens recommendation network\footnote{https://movielens.org/}. The schema of these networks is depicted in Fig.~\ref{fig:schema}. As an example, in the bibliographic network, $\mc{V}=\left\lbrace {Author}, {Paper}, {Venue}, {Term}\right\rbrace$ is the set of different node types, and $\mc{E}=\left\lbrace\text{write}, \text{publish}, \text{mention}, \text{cite}\right\rbrace$ is the set of different link types.

{\color{red}
Analogous to homogeneous networks where we use an adjacency matrix to represent whether pairs of nodes are linked to each other or not, in heterogeneous networks, we define \emph{Heterogeneous Adjacency Matrices} to represent the connectivity of nodes of different types:

\begin{definition}{(Heterogeneous Adjacency Matrix)}
Given a heterogeneous network $G$ with schema $\mc{S}_G=(\mc{V}, \mc{E})$, for each link type $\varepsilon\in\mc{E}$ denoting the relation between node types $\nu_i,\nu_j\in\mc{V}$, the heterogeneous adjacency matrix $M_{\varepsilon}$ is a binary matrix representing whether nodes of type $\nu_i$ are in relation with nodes of type $\nu_j$ with link type $\varepsilon$ or not.
\end{definition}

For instance, in the bibliographic network, the heterogeneous adjacency matrix $M_{write}$ is a binary matrix where each row is associated to an author and each column is associated to a paper, and $M_{write}{(i,j)}$ indicates if the author $i$ has written the paper $j$.
}

As we mentioned in the Introduction section about heterogeneous networks, the concept of a link can be generalized to a relationship. In this case, a relationship could be either a single link, or a composite relation constituted by concatenation of multiple links that together have a particular semantic meaning. For example, the co-authorship relation in the bibliographic network with the schema shown in Fig.~\ref{fig:schema:dblp}, can be defined as the combination of two \emph{Author-{write}-Paper} links, making \emph{Author-{write}-Paper-{write}-Author} relation. When dealing with link or relationship prediction in heterogeneous networks, we must exactly specify what kind of link or relationship we are going to predict. This specific relation to be predicted is called the \emph{Target Relation} \cite{sun2012will}. For example, in DBLP bibliographic network we aim to predict if and when an author will cite a paper from another author. Thus the target relation in this case would be \emph{Author-{write}-Paper-{cite}-Paper-{write}-Author}.

\subsection{Dynamic Information Networks}
An information network is \emph{dynamic} when its nodes and linkage structure can change over time. That is, in a dynamic information network, all nodes and links are associated with a birth and death time. More formally, a dynamic network at the timestamp $\tau$ is defined as $G^{\tau}=(V^{\tau}, E^{\tau})$ where $V^{\tau}$ and $E^{\tau}$ are respectively the set of nodes and the set of links existing in the network at the timestamp $\tau$.
% , which are defined as follows:
% \[V^{t_0,t_1}=\{v: v\in V \land t_0< t_b(v)\le t_1<t_d(v)\}\]
% \[E^{t_0,t_1}=\{e: e\in E \land t_0< t_b(e)\le t_1<t_d(e)\}\]
% where $V$ and $E$ are accordingly the set of all nodes and links ever appeared or removed during the time interval $(t_0,t_1]$, and $t_b(x)$ and $t_d(x)$ denotes the birth and death time of a network entity $x$ (which can be node or link), respectively. For the sake of simplicity, we assume that each network entity have only a single birth and death time, and we do not consider the cases of reoccurring or multiple links.
%Furthermore, without loss of generality, we assume that the node objects are persistent over time, and only linkage structure would change. This is due to the fact that an isolated node which does not participate in any link will not have any effect on the network and can be simply ignored. By this definition, a node will bear when it make a link with another node for the first time, and it dies when it loses all of its associated links. This can be formally stated as below:
%\[\forall v\in V^{0,\infty}: t_b(v)=\min_e t_b(e), \ v\in e\in E^{0,\infty}\]
%\[\forall v\in V^{0,\infty}: t_d(v)=\max_e t_d(e), \ v\in e\in E^{0,\infty}\]

In this paper, we consider the case that an information network is both dynamic and heterogeneous. This means that all network entities are associated with a type, and can possibly have birth and death times, regardless of their types. The bibliographic network is an example of both dynamic and heterogeneous one. Whenever a new paper is published, a new \emph{Paper} node will be added to the network, alongside with the corresponding new \emph{Author}, \emph{Term}, and \emph{Venue} nodes (if they don't exist yet). New links will be formed among these newly added nodes to indicate the \textit{write}, \textit{publish} and \textit{mention} relationships. Some linkages might also form between the existing nodes and the new ones, like new \textit{cite} links connecting the new paper with the existing papers in its reference list.

{\color{red}In order to formally describe the state of a heterogeneous and dynamic network at any timestamp $\tau$, we define \emph{time-aware heterogeneous adjacency matrix} as bellow:
	
	\begin{definition}{(Time-Aware Heterogeneous Adjacency Matrix)}
		Given a dynamic heterogeneous network $G^\tau$ with schema $\mc{S}_G=(\mc{V}, \mc{E})$, for each link type $\varepsilon\in\mc{E}$ denoting the relation between node types $\nu_i,\nu_j\in\mc{V}$, the time-aware heterogeneous adjacency matrix $M^\tau_{\varepsilon}$ is a binary matrix representing if nodes of type $\nu_i$ are in relation with nodes of type $\nu_j$ with link type $\varepsilon$ at the timestamp $\tau$. More formally, for $a\in\nu_i$ and $b\in\nu_j$ we have:
		\[M^\tau_{\varepsilon}(a,b)=\begin{cases} 
		1, & \text{if}\quad(a,b)\in\varepsilon\quad\text{and}\quad bt(a,b) < \tau \le dt(a,b) \\
		0, & \text{otherwise}
		\end{cases}
		\]
		where $bt(a,b)$ and $dt(a,b)$ denote the birth and the death time of the link $(a,b)$, respectively.
	\end{definition}
}

\subsection{Continuous-Time Relationship Prediction}
Suppose that we are given a dynamic and heterogeneous information network as $G^{\tau}$ lastly observed at the timestamp $\tau$, together with its network schema $S_G$. Now, given the target relation $R$, the aim of continuous-time relationship prediction is to forecast the building time $t\ge \tau$ of the target relation $R$ between any node pair $(a,b)$ in $G^{\tau}$.

{\color{red}
	In order to solve this problem, we aim to train a supervised model that given a pair of nodes like $(a,b)$, it can predict a point estimate on the time it takes for the relationship of type $R$ to be formed between them. The input to such model will be a feature vector $\mb{x}$ corresponding to the node pair $(a,b)$. The model will then output with a continuous variable $t$ that indicates when the relationship of type $R$ will be built between $a$ and $b$. In order to train such model, we need to assemble a dataset comprising the feature vectors of all the node pairs between which the relation $R$ have already been formed. The process of selecting sample node pairs, extracting their feature vector, and training the supervised model are what we cover in the subsequent sections. 
}
%In the next section, we introduce a feature extraction framework to cope with both dynamicity and heterogeneity of the network built upon the concepts and definitions provided earlier. Then in Section~\ref{sec:method}, we propose a non-parametric model which utilizes the extracted features to perform predictions about the building time of the target relationship.



