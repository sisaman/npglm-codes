\begin{abstract}
Online social networks, World Wide Web, media and technological networks, and other types of so-called \emph{information networks} are ubiquitous nowadays. These information networks are inherently \emph{heterogeneous} and \emph{dynamic}. They are heterogeneous as they consist of multi-typed objects and relations, and they are dynamic as they are constantly evolving over time. One of the challenging issues in such heterogeneous and dynamic environments, is to forecast those relationships in the network that will appear in the future. In this paper, we try to solve the problem of continuous-time relationship prediction in dynamic and heterogeneous information networks. This implies predicting the time it takes for a relationship to appear in the future, given its features that have been extracted by considering both the heterogeneity and the temporal dynamics of the underlying network. To this end, we first introduce a meta-path-based feature extraction framework to effectively extract features suitable for relationship prediction regarding the heterogeneity and dynamicity of the network. Next, we propose a supervised non-parametric approach, called \emph{Non-Parametric Generalized Linear Model} (\npglm), which infers the hidden underlying probability distribution of the relationship building time given its features. We then present a learning algorithm to train \npglm and an inference method to answer time-related queries. Extensive experiments conducted on both synthetic dataset and real-world DBLP bibliographic citation network dataset demonstrate the effectiveness of \npglm in solving continuous-time relationship prediction problem vis-\`a-vis alternative baselines.
\end{abstract}