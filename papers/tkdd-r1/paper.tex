\pdfoutput=1
\documentclass[format=acmsmall, review=false, screen=true]{acmart}

\usepackage{bm}
\usepackage{booktabs} % For formal tables
\usepackage{pgfplots}
\usepackage{xspace}
\usepackage{subfig}
\usepackage{chronosys}
\usepackage{enumerate}
\usepackage[inline]{enumitem}
\usepackage{tabu}
\usepackage{multirow}
\usepackage[ruled]{algorithm2e}
\usepackage{hyperref}           %5
\hypersetup{
	colorlinks=true,
	linkcolor=blue,
	filecolor=red,      
	urlcolor=magenta,
	breaklinks=true,            %3
}
\usepackage{breakurl}           %3
\pgfplotsset{compat=1.13}
\usetikzlibrary{arrows,positioning,decorations,shapes}

\SetAlFnt{\small}
\SetAlCapFnt{\small}
\SetAlCapNameFnt{\small}
\SetAlCapHSkip{0pt}
\IncMargin{-\parindent}


% Metadata Information
\acmJournal{TKDD}
%\acmVolume{9}
%\acmNumber{4}
\acmArticle{1}
\acmYear{2018}
%\acmMonth{3}
%\copyrightyear{2009}
%\acmArticleSeq{9}

% Copyright
\setcopyright{none}
%\setcopyright{acmcopyright}
%\setcopyright{acmlicensed}
%\setcopyright{rightsretained}
%\setcopyright{usgov}
%\setcopyright{usgovmixed}
%\setcopyright{cagov}
%\setcopyright{cagovmixed}

% DOI
%\acmDOI{0000001.0000001}

% Paper history
%\received{February 2007}
%\received[revised]{March 2009}
%\received[accepted]{June 2009}

\settopmatter{printacmref=false}


\begin{document}
\title[Continuous-Time Relationship Prediction in Dynamic Heterogeneous Networks]{Continuous-Time Relationship Prediction in Dynamic Heterogeneous Information Networks}
%\titlenote{Produces the permission block, and
%  copyright information}

\author{Sina Sajadmanesh}
\orcid{0000-0002-8834-0338}
\affiliation{%
	\institution{Department of Computer Engineering, Sharif University of Technology}
	\streetaddress{Azadi Ave}
	\city{Tehran}
	\state{Tehran}
	\postcode{1458889694}
	\country{Iran}}
\email{sajadmanesh@ce.sharif.edu}

\author{Sogol Bazargani}
\affiliation{%
	\institution{Department of Computer Engineering, Sharif University of Technology}
	\streetaddress{Azadi Ave}
	\city{Tehran}
	\state{Tehran}
	\postcode{1458889694}
	\country{Iran}}


\author{Jiawei Zhang}
\affiliation{%
	\institution{Department of Computer Science, Florida State University}
	\streetaddress{1017 Academic Way}
	\city{Tallahassee}
	\state{Florida}
	\postcode{FL 32304}
	\country{United States}}
\email{jzhang@cs.fsu.edu}

\author{Hamid R. Rabiee}
\affiliation{%
	\institution{Department of Computer Engineering, Sharif University of Technology}
	\streetaddress{Azadi Ave}
	\city{Tehran}
	\state{Tehran}
	\postcode{1458889694}
	\country{Iran}}
\email{rabiee@sharif.edu}


% The default list of authors is too long for headers.
\renewcommand{\shortauthors}{Sajadmanesh et al.}
\newcommand{\descr}[1]{\smallskip\noindent\textbf{#1}}
\newcommand{\npglm}{{\textsc{Np-Glm}}\xspace}
\newcommand{\mb}[1]{\mathbf{#1}}
\newcommand{\mc}[1]{\mathcal{#1}}
\newcounter{sarrow}
\newcommand\xrsquigarrow[1]{%
	\stepcounter{sarrow}%
	\begin{tikzpicture}[decoration=snake]
	\node (\thesarrow) {\strut#1};
	\draw[->,decorate] (\thesarrow.south west) -- (\thesarrow.south east);
	\end{tikzpicture}%
}

\begin{abstract}
Online social networks, World Wide Web, media and technological networks, and other types of so-called \emph{information networks} are ubiquitous nowadays. These information networks are inherently \emph{heterogeneous} and \emph{dynamic}. They are heterogeneous as they consist of multi-typed objects and relations, and they are dynamic as they are constantly evolving over time. One of the challenging issues in such heterogeneous and dynamic environments is to forecast those relationships in the network that will appear in the future. In this paper, we try to solve the problem of continuous-time relationship prediction in dynamic and heterogeneous information networks. This implies predicting the time it takes for a relationship to appear in the future, given its features that have been extracted by considering both heterogeneity and temporal dynamics of the underlying network. To this end, we first introduce a feature extraction framework that combines the power of meta-path-based modeling and recurrent neural networks to effectively extract features suitable for relationship prediction regarding heterogeneity and dynamicity of the networks. Next, we propose a supervised non-parametric approach, called \emph{Non-Parametric Generalized Linear Model} (\npglm), which infers the hidden underlying probability distribution of the relationship building time given its features. We then present a learning algorithm to train \npglm and an inference method to answer time-related queries. Extensive experiments conducted on synthetic data and three real-world datasets, namely Delicious, MovieLens, and DBLP, demonstrate the effectiveness of \npglm in solving continuous-time relationship prediction problem vis-\`a-vis competitive baselines. %\footnote{Codes and data are available at \url{https://github.com/sisaman/npglm}}.
\end{abstract}

%
% The code below should be generated by the tool at
% http://dl.acm.org/ccs.cfm
% Please copy and paste the code instead of the example below.
%
\begin{CCSXML}
	<ccs2012>
	<concept>
	<concept_id>10002951.10003227.10003351</concept_id>
	<concept_desc>Information systems~Data mining</concept_desc>
	<concept_significance>500</concept_significance>
	</concept>
	<concept>
	<concept_id>10002951.10003260.10003261.10003270</concept_id>
	<concept_desc>Information systems~Social recommendation</concept_desc>
	<concept_significance>500</concept_significance>
	</concept>
	<concept>
	<concept_id>10010147.10010257</concept_id>
	<concept_desc>Computing methodologies~Machine learning</concept_desc>
	<concept_significance>300</concept_significance>
	</concept>
	</ccs2012>
\end{CCSXML}

\ccsdesc[500]{Information systems~Data mining}
\ccsdesc[500]{Information systems~Social recommendation}
\ccsdesc[300]{Computing methodologies~Machine learning}

%
% End generated code
%


\keywords{Link Prediction, Social Network Analysis, Heterogeneous Network, Non-Parametric Modeling, Recurrent Neural Network, Autoencoder}


\maketitle

\section{Introduction}\label{sec:intro}
Link prediction is the problem of prognosticating a certain relationship, like interaction or collaboration, between two entities in a networked system \cite{liben2007link}. It has a long literature and is studied extensively in the last decade. Initial works on link prediction problem mostly concentrate on homogeneous networks, which are composed of single type of nodes connected by links of the same type. However, many of today's networks, such as bibliographic or online social networks, are inherently \emph{heterogeneous}, in which multiple types of nodes are interconnected using multiple types of links \cite{shi2017survey}. For example, a bibliographic network contains author, paper, venue, etc. as different node types; and write, publish, cite, and so on as diverse link types that bind different node types to each other. 
%In these heterogeneous networks, the concept of a link can be generalized to a relationship, which can be constructed by combining different links with different types. For instance, author-cite-paper relationship can be defined in a bibliographic network as a combination of author-write-paper and paper-cite-paper links. Analogously, one can generalize the link prediction to \emph{relationship prediction} in heterogeneous networks which tries to predict complex relationships instead of links \cite{sun2012will}.

While most of the studies on the link prediction in heterogeneous networks utilize a static snapshot of the underlying network, many of these networks are \emph{dynamic} in nature, which means that new nodes and linkages are continually added to the network, and some existing nodes and links may be removed from the network over time. For example, in online social networks, such as Facebook, new users are joining in the network every day, and new friendship links are being added to the network gradually. This dynamic characteristic causes the structure of the network to change and evolve over time, and taking these changes into account can significantly boost the quality of link prediction task \cite{potgieter2009temporality}.

In recent years, newer studies have shifted from traditional link prediction on static and homogeneous networks toward newer domains, considering heterogeneity and dynamicity of networks \cite{dong2012link, davis2011multi}. However, most of these works merely focus on one of these aspects, disregarding the other. Although there are quite a few studies that address both the challenges of heterogeneity and dynamicity \cite{aggarwal2012dynamic, sett2017temporal}, to the best of our knowledge, all of them have ultimately formulated the link prediction problem as a binary classification task, i.e., predicting \emph{whether} a link will appear in the network in the future. However, in dynamic networks, new links are continually appearing over time. So a much more interesting problem, which we call it \emph{continuous-time link prediction} in this paper, is to predict \emph{when} a link will emerge or appear between two nodes in the network. Examples of this problem include predicting the time at which two individuals become friends in a social network, or the time when two authors collaborate on writing a paper in a bibliographic network \cite{sun2012will}. Inferring the link formation time in advance can be very useful in many concrete applications. For example, if a social network recommender system could predict the link formation time between two people, then it can issue a friendship suggestion close to that time since it will have a relatively higher chance to be accepted. Additionally in biological context, predicting the marker proteins interaction time in a gene regulatory network will lead to predicting tumor progression and prognosis \cite{taylor2009dynamic}.
%Good continuous-time link prediction results will lead to denser connections among users, and can greatly improve users' engagement that is the ultimate goal of online social networks \cite{kwak2010twitter}.

In this paper, we aim to solve the problem of continuous-time link prediction, in which we forecast the link formation time between two nodes in a dynamic and heterogeneous environment. This problem is very challenging from the technical perspective, and cannot be solved trivially for three main reasons. First, the formulation of continuous-time link prediction is quite different from the conventional link prediction due to the involvement of temporal dynamics of the network and the necessity of considering network evolution time-line. Second, we only know the formation time of those links that are already present at the network and for the rest of them that are yet to happen, which are excessive in number versus the existing ones, we lack such information. Finally, as opposed to the works concerning the binary link prediction, there are very rare works in the literature on continuous-time link prediction that attempt to answer the ``when'' question. To the best of our knowledge, the only work that has studied the continuous-time link prediction problem so far, is proposed by Sun et al. \cite{sun2012will}. They infer a probability distribution over time for each pair of nodes given their features, and answer time-related queries about the link formation time between the two nodes using the inferred distribution. However, the drawback of their method, not to mention neglecting the temporal dynamics of the network, is that it mainly relies on the assumption that link formation times are coming from a certain probability distribution that must be fixed beforehand. This assumption though simplifying is very restrictive, because in real applications this distribution is unknown, and considering any specific one as a priori could be far from reality or limit the solution generality.

In order to address the above challenges, we propose a supervised non-parametric method to solve the problem of continuous-time link prediction. To this end, we first formally define the continuous-time link prediction problem and formulate the approach to solve it generally. Then, we introduce our novel feature extraction framework which leverages meta-path-based modeling and recurrent neural networks to deal with heterogeneity and dynamicity of information networks. Next, we present \emph{Non-Parametric Generalized Linear Model} (\npglm) which models the distribution of link formation time given the extracted features. The strength of this non-parametric model is that it is capable of learning the underlying distribution of the link formation time.
%Afterwards, we propose an inference algorithm to answer queries, like the most probable time by which a relationship will appear between two nodes, or the probability of relationship creation between them during a specific period. 
Finally, we conduct comprehensive experiments over two real-world datasets - Delicious and MovieLens - to demonstrate the effectiveness and generality of the proposed method in predicting link formation time versus relevant baselines. As a summary, we can enumerate our major contributions as follows:

\setlist[enumerate]{itemsep=0mm}
\begin{enumerate}[label=(\roman*),wide]
\item The proposed feature extraction framework can tackle heterogeneity of the data as well as capturing the temporal dynamics of the network by incorporating meta-path-based features into a recurrent neural network based autoencoder.
\item Our non-parametric model takes a unique approach toward learning the underlying distribution of link formation time without imposing any significant assumptions.
\item Extensive evaluations over real-world datasets are performed to investigate the effectiveness of the proposed method. 
\item To the best of our knowledge, this paper is the first one which studies the continuous-time link prediction problem in both dynamic and heterogeneous networks.
\end{enumerate}

The rest of this paper is organized as follows. In the next section, we introduce our novel feature extraction framework. Then, we go through the details of the proposed \npglm method in Section \ref{sec:method}. Experimental results are described in Section \ref{sec:results}. Section \ref{sec:related} discusses the related works, and finally in Section \ref{sec:conclusion}, we conclude the paper.

\section{Problem Formulation}\label{sec:problem}
In this section, we formulate the temporal link prediction problem and introduce some important concepts and definitions used throughout the paper.

\begin{table*}
%\renewcommand{\arraystretch}{2}
\centering
\caption{Characteristics of Some Probability Distributions Used for Event-Time Modeling}
\label{table:dists}
\begin{tabu} to \textwidth {X X[c] X[c] X[c] X[c]}
\toprule
Distribution & Density function & Reliability function & Intensity function & Cumulative intensity\\
& $f_T(t)$ & $S(t)$ & $\lambda(t)$ & $\Lambda(t)$\\[1pt]
\midrule % In-table horizontal line
Exponential & $\alpha\exp(-\alpha t)$ & $\exp(-\alpha t)$ & $\alpha$ & $\alpha t$\\[4pt]
%\midrule
Rayleigh & $\frac{t}{\sigma^2}\exp(-\frac{t^2}{2\sigma^2})$ & $\exp(-\frac{t^2}{2\sigma^2})$ & $\frac{t}{\sigma^2}$ & $\frac{t^2}{2\sigma^2}$\\[4pt]
%\midrule % In-table horizontal line
Gompertz & $\alpha e^t\exp\left\lbrace -\alpha(e^t-1) \right\rbrace$ & $\exp\left\lbrace -\alpha(e^t-1) \right\rbrace$ & $\alpha e^t$ & $\alpha e^t$\\[2pt]
%\midrule % In-table horizontal line
Power-Law & $\frac{\alpha\beta^\alpha}{t^{\alpha+1}}$ & $\left(\frac{\beta}{t}\right)^\alpha$ & $\frac{\alpha}{t}$ & $\alpha\ln(t)$\\
\bottomrule % Bottom horizontal line
\end{tabu}
\end{table*}

\subsection{Temporal Link Prediction}
The aim of this paper is to predict the time of link creation in social networks.
Formally, given the feature vector $x_l$ for a missing link $l$ extracted in time $t_0$, we want to predict $t_l$, which shows how long after $t_0$ the link $l$ will appear in the network. A probabilistic approach to this problem is to model the conditional distribution $f_T(t_l\mid x_l)$.

\subsection{Data Description}
Suppose that we have a snapshot of the network at the time $t_0$, and we have seen the evolution of the network (the emergence of new links) in the time interval $[t_0,t_e]$ called \textit{time window}. Based on the existence state of the links prior to $t_0$, between $t_0$ and $t_e$, and after $t_e$, we can classify links in the following categories:

\begin{enumerate}
\item Links that are already present at time $t_0$.
\item Links that do not exist at $t_0$, but will appear during the time window.
\item Links that remain missing all the time when we reach $t_e$.
\end{enumerate}

Those links that fall within the 2nd and the 3rd categories form our data samples and will be used in the learning procedure. For these links, we extract their feature vectors at time $t_0$. For a link $l$ of the 2nd category, we have seen that it is created at a time like $t_c\in[t_0,t_e]$. So we set $t_l=t_c-t_0$ as the time it takes for the link $l$ to appear after $t_0$, and $y_l=1$ which indicates that we have \emph{observed} its exact creation time. If $l$ is of the 3rd category, we haven't seen its exact creation time, but we know it is definitely after $t_e$. For such samples, which we call the \emph{censored} ones, we set $t_l=t_e-t_f$ and $y_l=0$ to indicate that the recorded time is in fact less than the real one. These type of links are also of interest because their features will give us some information about their time falling after $t_e$. As a result, each link $l$ is associated with a triple $(x_l,y_l,t_l)$ representing its feature vector, its observation status, and the time it takes to appear, respectively. In Section \ref{sec:method}, we propose \npglm which is a supervised method to relate $x_l$ to $t_l$ by estimating $f_T(t_l\mid x_l)$ in a non-parametric fashion.

\subsection{Basic Concepts}
Here we define some essential concepts that are necessary to study before we proceed to the proposed method. Generally, the formation of a link between two nodes in the network can simply be considered as an event with its occurring time as a random variable $T$ coming from a density function $f_T(t)$. Regarding this, we can have the following definitions:

\begin{definition}[Survival Function]
Given the density $f_T(t)$, the survival function denoted by $S(t)$, is the probability that an event occurs after a certain value of $t$, which means:
\begin{equation}
    S(t) = P(T > t) = \int_t^\infty f_T(t)dt
\end{equation}
\end{definition}

\begin{definition}[Intensity Function]
The intensity function (or failure rate function), denoted by $\lambda(t)$, is the instantaneous rate of event occurring at any time $t$ given the fact that the event has not occurred yet:
\begin{equation}
    \lambda(t)=\lim_{\Delta t\rightarrow 0}\frac{P(t\le T\le t+\Delta t\mid T\ge t)}{\Delta t}
\end{equation}
\end{definition}

\begin{definition}[Cumulative Intensity Function]
The cumulative intensity function, denoted by $\Lambda(t)$, is the area under the intensity function up to a point $t$:
\begin{equation}
    \Lambda(t)=\int_0^t\lambda(t)dt
\end{equation}
\end{definition}

The relations between density, survival, and intensity functions come directly from their definitions as follows:

\begin{equation}\label{eq:intensity}
    \lambda(t)=\frac{f_T(t)}{S(t)}
\end{equation}
\begin{equation}\label{eq:reliability}
    S(t)=\exp(-\Lambda(t)dt)
\end{equation}
\begin{equation}\label{eq:density}
    f_T(t)=\lambda(t)\exp(-\Lambda(t))
\end{equation}

Table \ref{table:dists} shows the density, reliability, intensity, and cumulative intensity functions of some widely-used distributions for event time modeling.
\section{Feature Extraction Framework}\label{sec:features}

In this section, we present our framework to extract features which is designed to have three major characteristics: First, it effectively considers different type of nodes and links available in a heterogeneous information network and regards their impact on the building time of the target relationship. Second, it takes the temporal dynamics of the network into account and leverages the network evolution history instead of simply aggregating it into a single snapshot. Finally, the extracted features are suitable for not only the link prediction problem, but also the generalized \emph{relationship prediction}. We will incorporate these features in the proposed non-parametric model in Section~\ref{sec:method} to solve the continuous-time relationship prediction problem.

\begin{figure}
	\definecolor{blue}{HTML}{84CECC}
	\definecolor{darkblue}{HTML}{375D81}
	\definecolor{green}{HTML}{3F7F47}
	\begin{chronology}[align=left, startyear=0,stopyear=200, width=\columnwidth, height=1pt, startdate=false, stopdate=false, arrowwidth=4pt, arrowheight=3pt]
		\footnotesize
		\chronoevent[date=false]{10}{$t_0$}
		\chronoevent[date=false]{40}{$t_0+\Delta$}
		\chronoevent[date=false]{70}{$t_0+2\Delta$}
		\chronoevent[date=false,mark=false]{100}{$\dots$}
		\chronoevent[date=false]{130}{$t_0+k\Delta$}
		\chronoevent[date=false]{190}{$t_1$}
		\chronoperiode[color=darkblue, startdate=false, bottomdepth=2pt, topheight=5pt, textdepth=8pt, stopdate=false]{10}{40}{$\Delta$}
		\chronoperiode[color=blue, startdate=false, bottomdepth=10pt, topheight=15pt, textdepth=-15pt, stopdate=false]{10}{129}{Feature Extraction Window $(\Phi=k\Delta)$}
		\chronoperiode[color=green, startdate=false, bottomdepth=10pt, topheight=15pt, textdepth=-15pt, stopdate=false]{131}{190}{Observation Window $(\Omega)$}
	\end{chronology}
	\caption{The evolutionary timeline of the network data.}
	\label{fig:timeline}
\end{figure}

\subsection{Data Preparation For Feature Extraction}
To solve the problem of continuous-time relationship prediction in dynamic networks, we need to pay attention to the temporal history of the network data from two different points of view. First, we have to mind the evolution history of the network for feature extraction, so that the extracted features reflect the changes made in the network over time. Second, we have to specify the exact relationship building time for each pair of nodes, because our goal is to propose a supervised method to predict a continuous variable, which in this case is the relationship building time. Hence, for each sample pair of nodes, we need a feature vector $\mb{x}$, associated with a target variable $t$ which indicates the building time of the target relationship between them.

Suppose that we have observed a dynamic network $G^{\tau}$ recorded in the interval $t_0 <\tau\le t_1$. According to Fig.~\ref{fig:timeline}, we split this interval into two parts: the first part for extracting the feature $\mb{x}$, and the second for determining the target variable $t$. We refer to the first interval as \emph{Feature Extraction Window} whose length is denoted by $\Phi$, and the second as \emph{Observation Window}, whose length is denoted by $\Omega$. Now, based on the existence in the observation window, target relationships fall within one of the following three different groups:

\begin{enumerate}
	\item Relationships that have already been formed before the beginning of the observation window (formed in the feature extraction window).
	\item Relationships that will be built in the observation window for the first time (not existing before).
	\item Relationships that will not be formed at all (neither in the feature extraction window nor in the observation window).
\end{enumerate}

Those pairs of nodes that act as the starting and ending nodes of the relationships in the 2nd and 3rd categories constitute our data samples, and will be used in the learning procedure. For such pairs, we extract their feature vector $\mb{x}$ using the history available in the feature extraction window. For each node pair in the 2nd category, we see that the target relationship between them has been created at a time like $t_r\in(t_0+\Phi,t_1]$. So we set $t=t_r-(t_0+\Phi)$ as the time it takes for the relationship to form since the beginning of the observation window. For these samples, we also set an auxiliary variable $y=1$ which indicates that we have \emph{observed} their exact building time. On the other hand, For node pairs in the 3rd category, we haven't seen their exact building time, but we know that it should be definitely after $t_1$. For such samples, that we call \emph{censored} samples, we set $t=t_1-(t_0+\Phi)$ that is equal to the length of the observation window $\Omega$, and set $y=0$ to indicate that the recorded time is in fact a lower bound on the true relationship building time. These type of samples are also of interest because their features will give us some information about their time falling after $t_1$. As a result, each final sample is associated with a triple $(\mb{x},y,t)$ representing its feature vector, observation status, and the time it takes for the target relationship to be formed, respectively.

%In Section \ref{sec:method}, we propose \npglm which is a supervised method to relate $x_l$ to $t_l$ by estimating $f_T(t_l\mid x_l)$ in a non-parametric fashion.

%Here is an toy example in a bibliographic network: Suppose that we have the data of the papers published between the years 1990 and 2010. For all papers, we have their authors, venue (where they are published), indexing terms, and their references. For each author pair, the goal is to predict by when one of them will cite another, if she has not done yet. To this end, we pick an intermediary year such as 2000 as pivot, and split the the data into two part. The paper that are published just before the year 2000 will belong to the feature extraction window, and the rest of the papers will fall within observation window. Now, for each pair of authors who did not cite each other in feature extraction window which 

\begin{table}[t]
	\centering
	\caption{Similarity Meta-Paths in Different Networks}
	\label{table:meta}
	\footnotesize
%	\setlength\tabcolsep{0pt}
	\begin{tabular} {c c l}
		\toprule
		Network & Meta-Path & Semantic Meaning \\
		\midrule
		\multirow{8}{*}{\rotatebox{90}{DBLP}} 
		&&\\
		& $A\rightarrow P\leftarrow A$ & Authors co-write a paper\\
		& $A\rightarrow P\leftarrow A\rightarrow P\leftarrow A$ & Authors have common co-author\\
		& $A\rightarrow P\leftarrow V\rightarrow P\leftarrow A$ & Authors publish in the same venue\\
		& $A\rightarrow P\rightarrow T\leftarrow P\leftarrow A$ & Authors use the same term\\
		& $A\rightarrow P\rightarrow P\leftarrow P\leftarrow A$ & Authors cite the same paper\\
		& $A\rightarrow P\leftarrow P\rightarrow P\leftarrow A$ & Authors are cited by the same paper\\
		&&\\
		\midrule
		\multirow{5}{*}{\rotatebox{90}{Delicious}} 
		&&\\
		& $U\leftrightarrow U\leftrightarrow U$ & Users have common contact\\
		& $U\rightarrow B\leftarrow U$ & Users post the same bookmark\\
		& $U\rightarrow B\rightarrow T\leftarrow B\leftarrow U$ & Users post bookmarks with the same tag\\
		&&\\
		\midrule
		\multirow{13}{*}{\rotatebox{90}{MovieLens}} 
		&&\\
		& $M\rightarrow A\leftarrow M$ & Movies share an actor\\
		& $M\rightarrow C\leftarrow M$ & Movies belong to the same country\\
		& $M\rightarrow D\leftarrow M$ & Movies have the same director\\
		& $M\rightarrow G\leftarrow M$ & Movies have the same genre\\
		& $M\rightarrow T\leftarrow M$ & Movies have the same tag\\
%		\cmidrule{2-3}
		& $U\rightarrow M\leftarrow U$ & Users rate common movie\\
		& $U\rightarrow M\rightarrow A\leftarrow M\leftarrow U$ & Users rate movies sharing an actor\\
		& $U\rightarrow M\rightarrow C\leftarrow M\leftarrow U$ & Users rate movies from the same country\\
		& $U\rightarrow M\rightarrow D\leftarrow M\leftarrow U$ & Users rate movies of the same director\\
		& $U\rightarrow M\rightarrow G\leftarrow M\leftarrow U$ & Users rate movies with the same genre\\
		& $U\rightarrow M\rightarrow T\leftarrow M\leftarrow U$ & Users rate movies with the same tag\\
		&&\\
		\bottomrule
	\end{tabular}
\end{table}

\subsection{Dynamic Feature Extraction}
In this part, we describe how to utilize the temporal history of the network in the feature extraction window in order to extract features for continuous-time relationship prediction problem. We first begin with the meta-path-based feature set for heterogeneous information networks, and then incorporate these features into a \emph{recurrent neural network based autoencoder} to exploit the temporal dynamics of the network as well. Hereby, we begin by defining the concept of meta-path \cite{sun2011pathsim}:

\begin{definition}[Meta-Path]
	In a heterogeneous information network, a meta-path is a directed path following the graph of the network schema to describe the general relations that can be derived from the network. Formally speaking, given a network schema $\mc{S}_G=(\mc{V}, \mc{E})$, the sequence $\nu_1\xrightarrow{\varepsilon_1}\nu_2\xrightarrow{\varepsilon_2}\dots\nu_{k-1}\xrightarrow{\varepsilon_{k-1}}\nu_k$ is a meta-path defined on $S_G$ where $\nu_i\in \mc{V}$ and $\varepsilon_i\in \mc{E}$.
\end{definition} 

Meta-paths are commonly used in heterogeneous information networks to describe multi-typed relations that have concrete semantic meanings. For example, in the bibliographic network whose schema is shown in Fig.~\ref{fig:schema:dblp}, we can define the co-authorship relation by the following meta-path:
\[Author\xrightarrow{write}Paper\xleftarrow{write}Author\]
or simply by $A\rightarrow P\leftarrow A$. Another example is the author citation relation, which in this paper is used as the target relation for DBLP network. It can be specified as:
\[Author\xrightarrow{write}Paper\xrightarrow{cite}Paper\xleftarrow{write}Author\]
abbreviated as $A\rightarrow P\rightarrow P\leftarrow A$.

Among the possible meta-paths that can be defined on a network schema, there are some that capture the similarity between two nodes. For example, the co-authorship meta-path $A\rightarrow P\leftarrow A$ in a bibliographic network creates a sense of similarity between two \emph{Author} nodes. These type of meta-paths, called \emph{similarity meta-paths}, are widely used to define topological features for link prediction problem in heterogeneous networks \cite{sun2011co, zhang2014meta, 7752228}. Table~\ref{table:meta} presents a number of similarity meta-paths that can be defined on DBLP, Delicious, and MovieLens networks to capture the heterogeneous similarity between different node types.

The concept of similarity meta-paths can be extended to define heterogeneous features suitable for relationship prediction problem, where we have a target relation. Here we follow the same approach as in \cite{sun2012will} which suggests the following three meta-path-based blocks to describe features for relationship prediction problem, given a target relation between two nodes of type $A$ and $B$:
\begin{enumerate}
	\small
	\item $A\xrsquigarrow{similarity}A\xrsquigarrow{target}B$
	\item $A\xrsquigarrow{target}B\xrsquigarrow{similarity}B$
	\item $A\xrsquigarrow{relation}C\xrsquigarrow{relation}B$
\end{enumerate}
where $\rightsquigarrow$ denotes a meta-path, with labels \emph{similarity} and \emph{target} denoting a similarity meta-path and the target relation, respectively. The \emph{relation} label denotes an arbitrary meta-path relating two nodes of possibly different types. The first block tells that there are some nodes of type $A$ similar to a single node of the same type that has made the target relationship with a node of type $B$. Therefore, those similar nodes may also form the target relation with the type $B$ node. An analogous intuition is behind the second block. For the third, it says that some nodes of type $A$ are in relation with some type $C$ nodes, which are themselves in relation with some nodes of type $B$. Hence, it is likely that type $A$ nodes form some relationships, such as the target relationship, with type $B$ nodes.

As an example in DBLP bibliographic network, for the target relation we use $A\rightarrow P\rightarrow P\leftarrow A$ as the meta-path denoting the author citation relation. In Addition, Paper-cite-Author ($P\rightarrow P\rightarrow A$) and Author-cite-Paper ($A\rightarrow P\rightarrow P$) are also used as the arbitrary relations, and the similarity meta-paths for DBLP network from Table~\ref{table:meta} are used to define the features for author citation relationship prediction.

After specifying the suitable meta-paths, we need a method to quantify them as features. Due to the dynamicity of the network, different links are emerging and vanishing from the network over time. Therefore, the quantifying method must handle this dynamicity. Here, we formally define \emph{Time-Aware Meta-Path-based Features}:

\begin{definition}[Time-Aware Meta-Path-based Feature]
	Suppose that we are given a dynamic heterogeneous network $G^{\tau}$ along with its network schema $\mc{S}_G=(\mc{V}, \mc{E})$, and a target Relation $A\rightsquigarrow B$. {\color{red}For a given pair of nodes $a\in A$ and $b\in B$, and a meta-path $\Psi=A\xrightarrow{\varepsilon_1}\nu_1\xrightarrow{\varepsilon_2}\dots\nu_{n-1}\xrightarrow{\varepsilon_{n}}B$ defined on $\mc{S}_G$, the time-aware meta-path-based feature at the timestamp $\tau$ is calculated as:
	\begin{equation*}
		f_{\Psi}^\tau(a,b)=\Big(\prod_{i=1}^{n}M^\tau_{\varepsilon_i}\Big)_{a,b}
	\end{equation*}}
\end{definition}

By using the above definition, we will be able to quantify the number of instances of any particular meta-path at any specific timestamp $\tau$. {\color{red}The calculation of the above matrix multiplication can be done efficiently due to the following reasons:
\begin{enumerate}
\item The heterogeneous adjacency matrices are sparse, thus we can hugely reduce the time complexity of each single matrix multiplication using fast sparse matrix multiplication algorithms, such as \cite{yuster2005fast}.
\item The meta-paths we use for features can be decoupled into the the building blocks where each block is simply a similarity meta-path. Therefore, we can calculate the adjacency matrix of each similarity meta-path and then multiply them together to get the final matrix, avoiding recalculation of previously computed products.
\item Computation time for the similarity meta-paths can also be saved, as they are usually symmetric (look for example at those presented in Table~\ref{table:meta}). Therefore, we can reduce the number of products by transposing the intermediate results. For instance, the adjacency matrix of the similarity meta-path $A\rightarrow P\leftarrow V\rightarrow P\leftarrow A$ is $X\cdot X^T$ where $X=M_{\text{write}}\cdot M_{\text{publish}}$, reducing the number of multiplications from three to two.
\end{enumerate}
}

So far we proposed a method to calculate the time-aware meta-path-based features, which is the number of path instances of a particular meta-path at the timestamp $\tau$. If we set this timestamp to the end of the feature extraction window, it is as though we are aggregating the whole network into a single snapshot observed at time $t_0+\Phi$. In order to avoid such an aggregation, we divide the feature extraction window into a sequence of $k$ contiguous intervals of a constant size $\Delta$, as shown in Fig.~\ref{fig:timeline}. By doing so, we intend to extract time-aware features in each sub-window that results in a multivariate time series containing the information about the temporal evolution of the topological features between any pair of nodes. With this in mind, we define \emph{Dynamic Meta-Path-based Time Series} as follows:

\begin{definition}[Dynamic Meta-Path-based Time Series]
	Suppose that we are given a dynamic heterogeneous network $G^{\tau}$ observed in a feature extraction window of size $\Phi$ ($t_0<\tau \le t_0+\Phi$), along with its network schema $\mc{S}_G=(\mc{V}, \mc{E})$ and a target relation $A\rightsquigarrow B$. Also suppose that the feature extraction window is divided into $k$ fragments of size $\Delta$. For a given pair of nodes $a\in A$ and $b\in B$ in $G^{t_0+\Phi}$, and a meta-path $\Psi$ defined on $\mc{S}_G$, the dynamic meta-path-based time series of $(a,b)$ is calculated as:
	\begin{equation*}
		x_{\Psi}^i(a,b)=f_{\Psi}^{t_0+i\Delta}(a,b) - f_{\Psi}^{t_0+(i-1)\Delta}(a,b)\quad\quad i=1\dots k
	\end{equation*}
\end{definition}

For each unique meta-path designed using the triple building blocks described before, we get a unique time series. For each time step, we put the corresponding values from all time series into a vector. Consequently, we get a multivariate time series where each time step is vector-valued. For example, if we have $d$ meta-paths $\Psi_1$ to $\Psi_d$, then each time step of the resulting time series will be of the form $\mb{x}^i=[x_{\Psi_1}^i,\dots,x_{\Psi_d}^i]^T$. Such multivariate time series reflect how topological features between two nodes change across different snapshots of the network. Based on the level of the network dynamicity, it can capture increasing/decreasing trends or even periodic/re-occurring patterns.

Now it's the time to convert this multivariate time series into a single feature vector so that we can use it as the input of our non-parametric model that is discussed in the next section. A trivial solution would be to stack all vectors of the multivariate time series into a single one, and feed our model with this single vector. However, this approach will result in a very high dimensional vector as the number of time steps increases, and can lead to difficulties in the learning procedure due to the curse of dimensionality. This is in contrast with our expectation that more time steps means more information about the history of the network and should result in a better prediction model. To overcome this problem, we combine the power of recurrent neural networks, especially Long Short Term Memory (LSTM) units \cite{hochreiter1997long}, which have proven to be very successful in handling time series and sequential data, with autoencoders \cite{bengio2009learning}, which are widely used to learn alternative representations of the data such that the learned representation can reconstruct the original input.

\begin{figure}
	\centering
	\footnotesize
	\tikzstyle{block} = [rectangle,draw=black,minimum width=0.5cm, minimum height=0.25cm]
	\tikzstyle{arrow} = [thick,->,>=stealth]
	\tikzstyle{label} = [rectangle]
	\begin{tikzpicture}
	\node[block] (e1) at (0,0) {};
	\node[block] (e2) at (1,0) {};
	\node[block,draw=none] (ed) at (2,0) {$\dots$};
	\node[block] (ek) at (3,0) {};
	
	\node[block] (dk) at (4,0) {};
	\node[block] (dk1) at (5,0) {};
	\node[block,draw=none] (dd) at (6,0) {$\dots$};
	\node[block] (d1) at (7,0) {};
	
	\node[label] (ie1) at (0,-1) {${x}^1$};
	\node[label] (ie2) at (1,-1) {${x}^2$};
	\node[label] (iek) at (3,-1) {${x}^k$};
	
	\node[label] (idk) at (4,-1) {$\mb{x}$};
	\node[label] (idk1) at (5,-1) {$\mb{x}$};
	\node[label] (id1) at (7,-1) {$\mb{x}$};
	
	\node[label] (oek) at (3,1) {$\mb{x}$};
	\node[label] (odk) at (4,1) {${x}^k$};
	\node[label] (odk1) at (5,1) {${x}^{k-1}$};
	\node[label] (od1) at (7,1) {${x}^1$};
	
	\draw [arrow] (ie1) -- (e1);
	\draw [arrow] (ie2) -- (e2);
	\draw [arrow] (iek) -- (ek);
	
	\draw [arrow] (idk) -- (dk);
	\draw [arrow] (idk1) -- (dk1);
	\draw [arrow] (id1) -- (d1);
	
	\draw [arrow] (ek) -- (oek);
	\draw [arrow] (dk) -- (odk);
	\draw [arrow] (dk1) -- (odk1);
	\draw [arrow] (d1) -- (od1);
	
	\draw [arrow] (e1) -- (e2);
	\draw [arrow] (e2) -- (ed);
	\draw [arrow] (ed) -- (ek);
	\draw [arrow] (ek) -- (dk);
	\draw [arrow] (dk) -- (dk1);
	\draw [arrow] (dk1) -- (dd);
	\draw [arrow] (dd) -- (d1);
	
	\end{tikzpicture}
	\caption{The architecture of the LSTM Autoencoder used for dynamic feature extraction. The learned representation of the $k^{\text{th}}$ stage is used as the dynamic feature $\mb{x}$.}
	\label{fig:autoencoder}
\end{figure}

Inspired by the work of Dai and Le on semi-supervised sequence learning \cite{dai2015semi}, we design a LSTM autoencoder which takes a multivariate time series as input, and tries to encode it to a latent representation, so that it can then predict the input time series from the learned vector. The architecture of such autoencoder is illustrated in Fig.~\ref{fig:autoencoder}. Both encoder and decoder are built using a LSTM to process sequential input of length $k$. The encoder LSTM takes the input sequence (the multivariate time series) step by step. The output of the $k$th step will be the encoded feature vector that we will use as the input to \npglm method. In the learning phase of the autoencoder, this vector will be repeated $k$ times and will be pushed into decoder LSTM to produce the input sequence in reverse order. Reversing the target sequence will make the optimization of the model easier, since it causes the decoder to revert back the changes made by the encoder to the input sequence. By using a proper loss function, we force the $i$th output of the decoder LSTM to be as close as possible to the $(k-i+1)$th input of the encoder LSTM.

The benefits of using a LSTM autoencoder is two-fold: (1) since the autoencoder can reconstruct the original time series, which reflects the temporal dynamics of the network, we get minimum information loss in the learned vector; and (2) as we can set the dimensionality of the encoded vector to any desired value, we can evade the curse of dimensionality. We explain our proposed non-parametric model in the next section that takes the learned representation as the feature vector $\mb{x}$ and attempts to predict the corresponding time $t$. 



%\begin{table*}
%	%\renewcommand{\arraystretch}{2}
%	\centering
%	\caption{Characteristics of Some Probability Distributions Used for Event-Time Modeling}
%	\label{table:dists}
%	\begin{tabu} to \textwidth {X X[c] X[c] X[c] X[c]}
%		\toprule
%		Distribution & Density function & Survival function & Intensity function & Cumulative intensity\\
%		& $f_T(t)$ & $S(t)$ & $\lambda(t)$ & $\Lambda(t)$\\[1pt]
%		\midrule % In-table horizontal line
%		Exponential & $\alpha\exp(-\alpha t)$ & $\exp(-\alpha t)$ & $\alpha$ & $\alpha t$\\[4pt]
%		%\midrule
%		Rayleigh & $\frac{t}{\sigma^2}\exp(-\frac{t^2}{2\sigma^2})$ & $\exp(-\frac{t^2}{2\sigma^2})$ & $\frac{t}{\sigma^2}$ & $\frac{t^2}{2\sigma^2}$\\[4pt]
%		%\midrule % In-table horizontal line
%		Gompertz & $\alpha e^t\exp\left\lbrace -\alpha(e^t-1) \right\rbrace$ & $\exp\left\lbrace -\alpha(e^t-1) \right\rbrace$ & $\alpha e^t$ & $\alpha e^t$\\[4pt]
%		Weibull & $\frac{\alpha t^{\alpha-1}}{\beta^\alpha}\exp\left\lbrace-(\frac{t}{\beta})^\alpha\right\rbrace$ & $\exp\left\lbrace-(\frac{t}{\beta})^\alpha\right\rbrace$ & $\frac{\alpha t^{\alpha-1}}{\beta^\alpha}$ & $(\frac{t}{\beta})^\alpha$\\[2pt]
%		%\midrule % In-table horizontal line
%		%Power-Law & $\frac{\alpha\beta^\alpha}{t^{\alpha+1}}$ & $\left(\frac{\beta}{t}\right)^\alpha$ & $\frac{\alpha}{t}$ & $\alpha\ln(t)$\\
%		\bottomrule % Bottom horizontal line
%	\end{tabu}
%\end{table*}

\section{Proposed Non-Parametric Model}\label{sec:method}
In this section we introduce our proposed model, called \emph{Non-Parametric Generalized Linear Model}, to solve the problem of continuous-time link prediction based on the extracted features. 
If we denote the link formation time by $t$ and its features by $\mb{x}$, our aim is to model the probability density function $f_T(t\mid \mb{x})$. A conventional approach to modeling this function is to fix a parametric distribution for $t$ (e.g. Exponential distribution) and then relate $\mb{x}$ to $t$ using a Generalized Linear Model \cite{sun2012will}. The major drawback of this approach is that we need to know the exact distribution of the link formation time, or at least, we could guess the best one that fits. The alternative way that we follow is to learn the shape of $f_T(t\mid \mb{x})$ from the data using a non-parametric solution.

In the rest of this section, we first bring the necessary theoretical backgrounds related to the concept, and then we go through the details of the proposed model.

\subsection{Background}
Here we define some essential concepts that are necessary to study before we proceed to the proposed model. Generally, the formation of a link between two nodes in a network can simply be considered as an event with its occurring time as a random variable $T$ coming from a density function $f_T(t)$. Regarding this, we can have the following definitions:

\begin{definition}[Survival Function]
	Given the density $f_T(t)$, the survival function denoted by $S(t)$, is the probability that an event occurs after a certain value of $t$, which means:
	\begin{equation}
	S(t) = P(T > t) = \int_t^\infty f_T(t)dt
	\end{equation}
\end{definition}

\begin{definition}[Intensity Function]
	The intensity function (or failure rate function), denoted by $\lambda(t)$, is the instantaneous rate of event occurring at any time $t$ given the fact that the event has not occurred yet:
	\begin{equation}
	\lambda(t)=\lim_{\Delta t\rightarrow 0}\frac{P(t\le T\le t+\Delta t\mid T\ge t)}{\Delta t}
	\end{equation}
\end{definition}

%\begin{definition}[Cumulative Intensity Function]
%	The cumulative intensity function, denoted by $\Lambda(t)$, is the area under the intensity function up to a point $t$:
%	\begin{equation}
%	\Lambda(t)=\int_0^t\lambda(t)dt
%	\end{equation}
%\end{definition}

The relationships between density, survival, and intensity functions come directly from their definitions as follows:

\begin{equation}\label{eq:intensity}
f_T(t)=\lambda(t){S(t)}
\end{equation}
\begin{equation}\label{eq:reliability}
S(t)=\exp(-\int_0^t\lambda(t)dt)
\end{equation}
%\begin{equation}\label{eq:density}
%f_T(t)=\lambda(t)\exp(-\Lambda(t))
%\end{equation}

%Table \ref{table:dists} shows the density, survival, intensity, and cumulative intensity functions of some widely-used distributions for event time modeling.

\subsection{Model Description}
Looking at Eq.~\ref{eq:intensity}, we see that the density function can be specified uniquely with its intensity function. Since the intensity function often has a simpler form than the density itself, if we learn the shape of the intensity function, then we can infer the entire distribution eventually. Therefore, we focus on learning the shape of the conditional intensity function $\lambda(t\mid \mb{x})$ from the data, and then accordingly infer the conditional density function $f_T(t\mid \mb{x})$ based on the learned intensity.
In order to reduce the hypothesis space of the problem and avoid the curse of dimensionality, we assume that $\lambda(t\mid \mb{x})$, which is a function of both $t$ and $\mb{x}$, can be factorized into two separate positive functions as the following:
\begin{equation}\label{eq:lambda}
\lambda(t\mid \mb{x})=g(\mb{w}^T\mb{x})h(t)
\end{equation}
where $g$ is a function of $\mb{x}$ which captures the effect of features via a linear transformation using coefficient vector $\mb{w}$ independent of $t$, and $h$ is a function of $t$ which captures the effect of time independent of $x$. This assumption, referred to as \emph{proportional hazards condition} \cite{breslow1975analysis}, holds in GLM formulations of many event-time modeling distributions. Our goal is now to fix the function $g$ and then learn both the coefficient vector $\mb{w}$ and the function $h$ from the training data. In order to do so, we begin with the likelihood function of the data as the following:

\begin{equation}
\prod_{i=1}^{N}f_T(t_i\mid \mb{x}_i)^{y_i}P(T\ge t_i\mid \mb{x}_i)^{1-y_i}\\
\end{equation}
The likelihood consists of the product of two parts: The first part is the contribution of those samples for which we have observed their exact formation time, in terms of their density function. The second part on the other hand, is the contribution of the censored samples, for which we use the probability of the formation time being greater than the recorded one. By applying Eq.~\ref{eq:intensity}, \ref{eq:reliability}, and \ref{eq:lambda}, the likelihood function becomes:
\begin{equation}
\prod_{i=1}^{N}\left[g(\mb{w}^T\mb{x}_i)h(t_i)\right]^{y_i}\exp\lbrace-g(\mb{w}^T\mb{x}_i)\int_{0}^{t_i}h(t)dt\rbrace
\end{equation}

Since we don't know the form of $h(t)$, we cannot directly calculate the integral appeared in the likelihood function. To deal with this problem, we treat $h(t)$ as a non-parametric function by approximating it with a piecewise constant function that changes just in $t_i$s. Therefore, the integral over $h(t)$, denoted by $H(t)$, becomes a series:
\begin{equation}\label{eq:cumh}
H(t_i)=\int_{0}^{t_i}h(t)dt \simeq \sum_{j=1}^{i}h(t_j)(t_j-t_{j-1})
\end{equation}
assuming samples are sorted by $t$ in increasing order, without loss of generality. The function $H(t)$ defined above plays an important role in both learning and inference phases. In fact, both the learning and inference phases rely on $H(t)$ instead of $h(t)$.
Replacing the above series in the likelihood and taking the logarithm, we end up with the following log-likelihood function:

\begin{equation}\label{eq:logl}
\begin{split}
\log\mathcal{L}
=\sum_{i=1}^{N}\Big\lbrace& y_i\left[\log g(\mb{w}^T\mb{x}_i) + \log h(t_i)\right]\\&-g(\mb{w}^T\mb{x}_i)\sum_{j=1}^{i}h(t_j)(t_j-t_{j-1})\Big\rbrace\\
\end{split}
\end{equation}

Maximizing this log-likelihood function relies on the choice of the function $g$. There are no particular limits on the choice of $g$ except that it must be a non-negative function. For example, both quadratic and exponential functions of $\mb{w}^T\mb{x}$ will do the trick. Here, we proceed with $g(\mb{w}^T\mb{x})=\exp(\mb{w}^T\mb{x})$ since it makes the log-likelihood a convex function with respect to $\mb{w}$. Subsequent equations can be derived for other choices of $g$ analogously. Setting the log-likelihood derivative with respect to $h(t_k)$ to zero yields a closed form solution for $h(t_k)$:
\begin{equation}\label{eq:h}
h(t_k)=\frac{y_k}{(t_k-t_{k-1})\sum_{i=k}^{N}\exp(\mb{w}^T\mb{x}_i)}
\end{equation}

By applying Eq.~\ref{eq:cumh}, we get the following for $H(t_i)$:
\begin{equation}\label{eq:H}
H(t_i)=\sum_{j=1}^{i}\frac{y_j}{\sum_{k=j}^{N}\exp(\mb{w}^T\mb{x}_k)}
\end{equation}
which depends on the vector $\mb{w}$. On the other hand, we cannot obtain a closed form solution for $\mb{w}$ from the log-likelihood function. Therefore, we turn to use Gradient-based optimization methods to find the optimal value of $\mb{w}$. The negative log-likelihood function with respect to $\mb{w}$, denoted by $NL(\mb{w})$ is as follows:

\begin{equation}\label{eq:nlw}
NL(\mb{w})=\sum_{i=1}^{N}\left\lbrace\exp(\mb{w}^T\mb{x}_i)H(t_i)-y_i\mb{w}^T\mb{x}_i\right\rbrace
\end{equation}
which depends on the function $H$. As the learning of both $\mb{w}$ and $H$ depends on each other, they should be learned collectively. Here, we use an iterative algorithm to learn $\mb{w}$ and $H$ alternatively. We begin with a random vector $\mb{w}^{(0)}$. Then in each iteration $\tau$, we first update $H^{(\tau)}$ via Eq.~\ref{eq:H} using $w^{(\tau-1)}$. Next, we optimize Eq.~\ref{eq:nlw} using the values of $H^{(\tau)}(t_i)$ to obtain $\mb{w}^{(\tau)}$. We continue this routine until convergence.

%\begin{algorithm}[t]
%	\small
%	\SetAlgoLined
%	\KwIn{$\mb{X}_{N\times d}=(\mb{x}_1,\dots\mb{x}_N)^T$ as $d$-dimensional feature vectors, $\mb{y}_{N\times1}$ as observation states, and $\mb{t}_{N\times1}$ as recorded times.}
%	\KwOut{Learned parameters $\mb{w}_{d\times1}$ and $\mb{H}_{N\times1}$.}
%	$converged\leftarrow False$\;
%	$threshold\leftarrow10^{-4}$\;
%	$\tau\leftarrow 0$\;
%	$\log\mathcal{L}^{(\tau)}=-\infty$\;
%	Initialize $\mb{w}^{(\tau)}$ with random values\;
%	\While{Not $converged$}{
%		$\tau\leftarrow\tau+1$\;
%		Use Eq.~\ref{eq:H} to obtain $\mb{H}^{(\tau)}$ using $\mb{w}^{(\tau-1)}$\;
%		Optimize Eq.~\ref{eq:nlw} to obtain $\mb{w}^{(\tau)}$ using $\mb{H}^{(\tau)}$\;
%		Use Eq.~\ref{eq:logl} to obtain $\log\mathcal{L}^{(\tau)}$ using $\mb{w}^{(\tau)}$ and $\mb{H}^{(\tau)}$\;
%		
%		\If{$\left\|\log\mathcal{L}^{(\tau)} - \log\mathcal{L}^{(\tau-1)}\right\| < threshold$}{
%			$converged\leftarrow True$\;
%		}
%	}
%	$\mb{w}\leftarrow \mb{w}^{(\tau)}$\;
%	$\mb{H}\leftarrow \mb{H}^{(\tau)}$\;
%	\caption{The learning algorithm of \npglm}
%	\label{alg:learning}
%\end{algorithm}


%\subsection{Convergence Analysis}
%To analyze the convergence of likelihood maximization, we look at the likelihood function close to the extreme point (which we denote with $w_e$). For the extreme point $\frac{\partial \log \mathcal{L}}{\partial w}$ vanishes so we have: 
%\begin{equation}\label{eq:ext}
%\sum_{i=1}^{i=N} y_ix_i=\sum_{i=1}^{i=N} exp(w_e^Tx_i)H(t_i)x_i
%\end{equation}
%Looking into the second derivative we have:
%\begin{equation}\label{eq:dif2}
%-\frac{\partial^2 \log \mathcal {L}}{\partial w^2}=\sum_{i=1}^{i=N} exp(w^Tx_i)H(t_i)x_ix_i^T
%\end{equation}
%So for points near the extreme point putting $w=w_e+\delta w$ (where $\delta w$ is small), neglecting the higher order terms with respect to $\delta w$ we have:
%\begin{equation}
%\log \mathcal{L}(\delta w)=\log \mathcal{L}(w_e)-\frac{1}{2}\delta w^TM\delta w + O(\delta w^3)
%\end{equation}
%Which is of a quadratic form with respect to $\delta w$ and $M$ is the second derivative (Equation \ref{eq:dif2}) measured at $w_e$. $M$ measures the curvature of $\log \mathcal{L}$ hyper-surface near the extreme point and it's spectral radius gives us a measure of how fast our maximization procedure converges to $w_e$. 
%Without going through the maximization process we can establish a bound on $M$ (and thus convergence) by noting that amongst the $x_i$ we can find one (denoting it with $x_s$) such that:
%\begin{equation}
%M>=(\sum_{i=1}^{i=N} exp(w_e^Tx_i)H(t_i)x_i) x_s^T
%\end{equation}
%Comparing with (Equation \ref{eq:ext}) we can write the RHS as:
%\begin{equation}
%M_e=(\sum_{i=1}^{i=N} y_ix_i) x_s^T
%\end{equation}
%So our likelihood function converges to the extremum at a rate faster than a quadratic form with $M_e$.


%\subsection{Inference Queries}
%In this part, we explain how to answer the common inference queries based on the inferred distribution $f_T(t\mid \mb{x})$. Suppose that we have learned the vector ${\mb{w}}$ and the function ${H}$ using the training samples $(\mb{x}_i, y_i, t_i),\ i=1\dots N$ following Algorithm~\ref{alg:learning}. Afterwards, for a testing link $R$ associated with a feature vector $\mb{x}_R$, the following queries can be answered:\\
%
%
%\subsubsection{Ranged Probability} What is the probability for the link $R$ to be formed between time $t_\alpha$ and $t_\beta$? This is equivalent to calculating $P(t_\alpha \le T \le t_\beta \mid \mb{x}_R)$, which by definition is:
%\begin{equation}\label{eq:ranged}
%\begin{split}
%P(t_\alpha\le T \le t_\beta \mid \mb{x}_R) = S(t_\alpha\mid \mb{x}_R) - S(t_\beta\mid \mb{x}_R)\\
%= \exp\{-g(\mb{w}^T\mb{x}_R){H}(t_\alpha)\} - \exp\{-g(\mb{w}^T\mb{x}_R){H}(t_\beta)\}
%\end{split}
%\end{equation}
%The problem here is to obtain the values of ${H}(t_\alpha)$ and ${H}(t_\beta)$, as $t_\alpha$ and $t_\beta$ may not be among $t_i$s of the training samples, for which ${H}$ is estimated. To calculate ${H}(t_\alpha)$, we find $k\in\{1,2,\dots,N\}$ such that $t_k\le t_\alpha < t_{k+1}$.
%Due to the piecewise constant assumption for the function $h$, we get:
%\begin{equation}\label{eq:inf1}
%{h}(t_\alpha)=\frac{{H}(t_\alpha)-{H}(t_k)}{t_\alpha-t_k}
%\end{equation} 
%On the other hand, since $h$ only changes in $t_i$s, we have:
%\begin{equation}\label{eq:inf2}
%{h}(t_\alpha)={h}(t_{k+1})=\frac{{H}(t_{k+1})-{H}(t_k)}{t_{k+1}-t_k}
%\end{equation}
%Combining Eq.~\ref{eq:inf1} and \ref{eq:inf2}, we get:
%\begin{equation}\label{eq:inf3}
%{H}(t_\alpha)={H}(t_k)+(t_\alpha-t_k)\frac{{H}(t_{k+1})-{H}(t_k)}{t_{k+1}-t_k}
%\end{equation}
%Following the similar approach, we can calculate ${H}(t_\beta)$, and then answer the query using Eq.~\ref{eq:ranged}. The dominating operation here is to find the value of $k$. Since we have $t_i$s sorted beforehand, this operation can be done using a binary search with $O(\log N)$ time complexity.\\
%
%\subsubsection{Quantile} By how long the target link $R$ will be formed with probability $\alpha$? This question is equivalent to find the time $t_\alpha$ such that $P(T \le t_\alpha\mid x_R)=\alpha$. By definition, we have:
%\begin{equation*}
%\begin{split}
%1-P(T \le t_\alpha\mid \mb{x}_R)=S(t_\alpha\mid \mb{x}_R)&=\exp\{-g(\mb{w}^T\mb{x}_R){H}(t_\alpha)\}=1-\alpha
%\end{split}
%\end{equation*}
%Taking logarithm of both sides and rearranging, we get:
%\begin{equation}\label{eq:inf4}
%{H}(t_\alpha)=-\frac{\log(1-\alpha)}{g(\mb{w}^T\mb{x}_R)}
%\end{equation}
%To find $t_\alpha$, we first find $k$ such that ${H}(t_k)\le{H}(t_\alpha)<{H}(t_{k+1})$.
%We eventually have $t_k\le t_\alpha < t_{k+1}$ since $H$ is a non-decreasing function due to non-negativity of the function $h$. Therefore, we again end up with Eq.~\ref{eq:inf3}, by rearranging which we get:
%\begin{equation}\label{eq:inf5}
%t_\alpha=(t_{k+1}-t_k)\frac{{H}(t_\alpha)-{H}(t_k)}{{H}(t_{k+1})-{H}(t_k)}+t_k
%\end{equation}
%By combining the Eq.~\ref{eq:inf4} and \ref{eq:inf5}, we can obtain the value of $t_\alpha$, which is the answer to the quantile query. It worth mentioning that if $\alpha=0.5$ then $t_\alpha$ becomes the median of the distribution $f_T(t\mid \mb{x}_R)$. Here again the dominant operation is to find the value of $k$, which due to the non-decreasing property of the function ${H}$ can be found using a binary search with $O(\log N)$ time complexity.
%
%%\descr{Random Sampling.}
%%Generating random samples from the inferred distribution can easily be carried out using the Inverse-Transform sampling algorithm. To pick a random sample from the inferred distribution $f_T(t\mid x)$, we first generate uniform random variable $u\sim Uniform(0,1)$. Then, we find $k$ such that $S(t_{k+1}\mid x)\leq u\le S(t_k\mid x)$. We output $t_{k+1}$ as the generated sample. Again, searching for the suitable value of $k$ is the dominant operation which can be undertaken via binary search with $O(\log N)$ time complexity.
%

\section{Synthetic Evaluations}\label{sec:synthetic}
We use synthetic data to verify the correctness of \npglm and its learning algorithm. Since \npglm is a non-parametric method, we generate synthetic data using various parametric models with previously known random parameters and evaluate how well \npglm can learn the parameters and the underlying distribution of the generated data.

\subsection{Experiment Setup}
We consider generalized linear models of two widely used distributions for event-time modeling, Rayleigh and Gompertz, as the ground truth models for generating synthetic data. Algorithm~\ref{alg:syn} is used to generate a total of $N$ data samples with $d$-dimensional feature vectors, consisting $N_o$ non-censored (observed) samples and remaining $N_c=N-N_o$ censored ones. For all synthetic experiments, we generate 10-dimensional feature vectors ($d=10$). We repeat every experiment 100 times and report the average results.

\begin{algorithm}[t]
    \small
    \SetAlgoLined
    \KwIn{The number of observed samples $N_o$, the number of censored samples $N_c$, the dimension of the feature vectors $d$, and the desired distribution $dist$}
    \KwOut{Synthetically generated data $\mb{X}_{N\times d}$, $\mb{y}_{N\times1}$, and $\mb{t}_{N\times1}$.}
    $N\leftarrow N_o+N_c$\;
    Draw a weight vector $\mb{w}\sim\mathcal{N}(0,\mb{I}_d)$, where $\mb{I}_d$ is the $d$-dimensional identity matrix\;
    Draw scalar intercept $b\sim\mathcal{N}(0,1)$\;
    \For{$i\leftarrow1$ to $N$}{
        Draw feature vector $\mb{x}_i\sim\mathcal{N}(0,\mb{I}_d)$\;
        Set distribution parameter $\alpha_i\leftarrow\exp(\mb{w}^T\mb{x}_i+b)$\;
        \uIf{$dist == Rayleigh$}{
            Draw $t_i\sim\alpha_i~t\exp\{-0.5\alpha_it^2\}$\;
        }
        \uElseIf{$dist == Gompertz$}{
            Draw $t_i\sim\alpha_i~e^t\exp\{-\alpha_i(e^t-1)\}$\;
        }
    }
    
    Sort pairs $(\mb{x}_i,t_i)$ by $t_i$ in ascending order\;
    
    \For{$i\leftarrow1$ to $N_o$}{
        $y_i\leftarrow1$\;
    }
    \For{$i\leftarrow(N_o+1)$ to $N$}{
        $y_i\leftarrow0$\;
    }
    \caption{Synthetic dataset generation algorithm.}
    \label{alg:syn}
\end{algorithm}


\subsection{Experiment Results}
\subsubsection{Convergence Analysis}
Since \npglm's learning is done in an iterative manner, we first analyze whether this algorithm converges as the number of iterations increases. We recorded the log-likelihood of \npglm, averaged over the number of training samples $N$ in each iteration. We repeated this experiment for $N\in\{1000,2000,3000\}$ with a fixed censoring ratio of 0.5, which means half of the samples are censored. The result is depicted in Fig.~\ref{fig:syn-cvg-n}. We can see that the algorithm successfully converges with a rate depending on the underlying distribution. For the case of Rayleigh, it requires about 100 iterations to converge but for Gompertz, this reduces to about 30. Also, we see that using more training data leads to achieving more log-likelihood as expected.



In Fig.~\ref{fig:syn-cvg-c}, we fixed $N=1000$ and performed the same experiment this time using different censoring ratios. According to the figure, we see that by increasing the censoring ratio, the convergence rate increases. This is because \npglm infers the values of $H(t)$ for all $t$ in the observation window. Therefore, as the censoring ratio increases, the observation window is decreased, so \npglm has to infer a fewer number of parameters, leading to a faster convergence. Note that as opposed to Fig.~\ref{fig:syn-cvg-n}, here a higher log-likelihood doesn't necessarily indicate a better fit, due to the likelihood marginalization we get by censored samples.


%Next, we analyzed the performance of \npglm in terms of the achieved log-likelihood on a separate test dataset by gradually increasing the number of training samples under different censoring ratios. We put away a number of 100K synthetically generated test data samples and trained \npglm with a training dataset of size ranging from 100 to 900. In each step, we calculated the average log-likelihood of the trained model on the test. We repeated this experiment using different censoring ratios. The result is depicted in Fig.~\ref{fig:syn-logl-n}. According to the figure, the log-likelihood achieved on the test dataset gradually rises with the increase in the number of training samples. That is because using more training samples could result in a better estimate of the parameters by which the test data samples are generated.
\subsubsection{Performance Analysis}
Next, we evaluated how good \npglm can infer the parameters used to generate synthetic data. To this end, we varied the number of training samples $N$ and measured the mean absolute error (MAE) between the learned weight vector $\hat{\mathbf{w}}$ and the ground truth. Fig.~\ref{fig:syn-mae-n} illustrates the result for different censoring ratios. It can be seen that as the number of training samples increases, the MAE gradually decreases. The other point to notice is that more censoring ratio results in a higher error due to the information loss we get by censoring.

In another experiment, we investigated whether censored samples are informative or not. For this purpose, we fixed the number of observed samples $N_o$ and changed the number of censored samples from 0 to 200. We measured the MAE between the learned $\mb{w}$ and the ground truth for $N_o\in\{200,300,400\}$. The result is shown in Fig.~\ref{fig:syn-mae-c}. It clearly demonstrates that adding more censored samples causes the MAE to dwindle up to an extent, after which we get no substantial improvement. This threshold is dependent on the underlying distribution. In this case, for Rayleigh and Gompertz it is about 80 and 120, respectively.


\subsubsection{Running Time Analysis}    
Finally, we assess the running time of \npglm's learning algorithm against the size of the training data when it becomes relatively large. To this end, we varied the number of samples from 10K to 100M and measured the average running time of the learning algorithm of \npglm on a single machine whose specification is reported in Table~\ref{table:pc}. Fig.~\ref*{fig:syn-time-n} depicts the result in log-log scale for Rayleigh and Gompertz distributions under different censoring ratios selected from the set $\{0.05, 0.25, 0.50\}$. It can be seen from the figure that the running time scales linearly with the number of training samples since the number of parameters to be inferred in \npglm as a non-parametric model depends on the size of the training data. The censoring ratio though negligible in scale can impact the running time of the algorithm, with more censoring ratio resulting in less running time. This is because higher censoring ratio reduces the observation window, which in turn reduces the number of parameters.

\begin{table}[t!]
    \centering
    \caption{PC Specification and Configuration}
    \label{table:pc}
    %    \footnotesize
    \begin{tabular} {c c}
        \toprule
        Operating System & Windows 10\\
        CPU & Intel Core i7 1.8 GHz\\
        RAM & 12 GB DDR III\\
        GPU & Nvidia GeForce GT 750\\
        Disk Type & SSD\\
        Programming Language & Python 3.6 \\
        \bottomrule % Bottom horizontal line
    \end{tabular}
\end{table}

\begin{figure}[t]
    \subfloat[Rayleigh distribution]{
        \begin{tikzpicture}[trim axis left, trim axis right]
        \begin{axis}
        [
        tiny,
        width=0.4\columnwidth,
        height=4.5cm,
        legend pos=south east,
        legend style={font=\scriptsize,nodes={scale=0.75, transform shape}},
        xmajorgrids,
        y tick label style={
            /pgf/number format/.cd,
            fixed,
            fixed zerofill,
            precision=1,
            /tikz/.cd
        },
        xlabel=$Iteration$,
        %xticklabel style={rotate=90},
        ylabel=$\log\mathcal{L}$,
        ylabel shift = -4 pt,
        ymax=2.5,
        ymin=1.2,
        xmin=0,
        xmax=200,
        %ytick={0.08,0.10,...,0.2},
        xtick={20,60,...,180},
        restrict x to domain=0:200,
        legend entries={${\tiny N=1000}$, $N=2000$, $N=3000$},
        ]
        \addplot[color=cyan,  thick, dashed] table{results/cvg_ray_1000.txt};
        \addplot[color=orange,ultra thick, dotted] table{results/cvg_ray_2000.txt};
        \addplot[color=purple,thick] table{results/cvg_ray_3000.txt};
        \end{axis}
        \end{tikzpicture}
    }\hfil
    \subfloat[Gompertz distribution]{
        \begin{tikzpicture}[trim axis left, trim axis right]
        \begin{axis}
        [
        tiny,
        width=0.4\columnwidth,
        height=4.5cm,
        legend pos=south east,
        legend style={font=\scriptsize,nodes={scale=0.75, transform shape}},
        xmajorgrids,
        y tick label style={
            /pgf/number format/.cd,
            fixed,
            fixed zerofill,
            precision=1,
            /tikz/.cd
        },
        xlabel=$Iteration$,
        %xticklabel style={rotate=90},
        ylabel=$\log\mathcal{L}$,
        ylabel shift = -4 pt,
        ymax=2.3,
        %ymin=0.06,
        xmin=0,
        xmax=60,
        %ytick={0.08,0.10,...,0.2},
        xtick={10,20,...,50},
        restrict x to domain=0:100,
        legend entries={$N=1000$, $N=2000$, $N=3000$},
        ]
        \addplot[color=cyan  ,thick, dashed] table{results/cvg_gom_1000.txt};
        \addplot[color=orange,ultra thick, dotted] table{results/cvg_gom_2000.txt};
        \addplot[color=purple,thick] table{results/cvg_gom_3000.txt};
        \end{axis}
        \end{tikzpicture}
    }
    \caption{Convergence of \npglm's average log-likelihood ($\log\mathcal{L}$) for different number of training samples ($N$). Censoring ratio has been set to 0.5.}
    \label{fig:syn-cvg-n}
\end{figure}
\begin{figure}[t]
    \subfloat[Rayleigh distribution]{
        \begin{tikzpicture}[trim axis left, trim axis right]
        \begin{axis}
        [
        tiny,
        width=0.4\columnwidth,
        height=4.5cm,
        legend pos=south east,
        legend style={font=\scriptsize,nodes={scale=0.75, transform shape}},
        xmajorgrids,
        y tick label style={
            /pgf/number format/.cd,
            fixed,
            fixed zerofill,
            precision=1,
            /tikz/.cd
        },
        xlabel=$Iteration$,
        %xticklabel style={rotate=90},
        ylabel=$\log\mathcal{L}$,
        ylabel shift = -8 pt,
        %ymax=0.2,
        ymin=-2,
        xmin=0,
        xmax=100,
        %ytick={0.08,0.10,...,0.2},
        xtick={10,30,...,90},
        restrict x to domain=0:200,
        legend entries={5\% censoring, 25\% censoring, 50\% censoring},
        ]
        \addplot[color=cyan  ,thick, dashed] table{results/cvg_ray_5.txt};
        \addplot[color=orange,ultra thick, dotted] table{results/cvg_ray_25.txt};
        \addplot[color=purple,thick] table{results/cvg_ray_50.txt};
        \end{axis}
        \end{tikzpicture}
    }\hfil
    \subfloat[Gompertz distribution]{
        \begin{tikzpicture}[trim axis left, trim axis right]
        \begin{axis}
        [
        tiny,
        width=0.4\columnwidth,
        height=4.5cm,
        legend pos=south east,
        legend style={font=\scriptsize,nodes={scale=0.75, transform shape}},
        xmajorgrids,
        y tick label style={
            /pgf/number format/.cd,
            fixed,
            fixed zerofill,
            precision=1,
            /tikz/.cd
        },
        xlabel=$Iteration$,
        %xticklabel style={rotate=90},
        ylabel=$\log\mathcal{L}$,
        ylabel shift = -4 pt,
        %ymax=0.2,
        %ymin=0.06,
        xmin=0,
        xmax=60,
        %ytick={0.08,0.10,...,0.2},
        xtick={10,20,...,50},
        restrict x to domain=0:60,
        legend entries={5\% censoring, 25\% censoring, 50\% censoring},
        ]
        \addplot[color=cyan  ,thick, dashed] table{results/cvg_gom_5.txt};
        \addplot[color=orange,ultra thick, dotted] table{results/cvg_gom_25.txt};
        \addplot[color=purple,thick] table{results/cvg_gom_50.txt};
        \end{axis}
        \end{tikzpicture}
    }
    \caption{Convergence of \npglm's average log-likelihood ($\log\mathcal{L}$) for different censoring ratios with 1K samples.}
    \label{fig:syn-cvg-c}
\end{figure}

%\begin{figure}[t]
%    \hfill  
%    \subfloat[Rayleigh distribution]{
%        \begin{tikzpicture}[trim axis left, trim axis right]
%        \begin{axis}
%        [
%        tiny,
%        width=0.4\columnwidth,
%        height=4.5cm,
%        legend pos=south east,
%        legend style={font=\scriptsize,nodes={scale=0.75, transform shape}},
%        grid,
%        y tick label style={
%            /pgf/number format/.cd,
%            fixed,
%            fixed zerofill,
%            precision=1,
%            /tikz/.cd
%        },
%        xlabel=$ N $,
%        ylabel=$\log\mathcal{L}$,
%        ylabel shift = -4 pt,
%        %xticklabel style={rotate=90},
%        %        ymax=0.35,
%        xmin=0,
%        xmax=1000,
%        %        ytick={0.05,0.10,...,0.35},
%        xtick={100,300,...,900},
%        restrict x to domain=0:900,
%        legend entries={0\% censoring, 5\% censoring, 10\% censoring},
%        ]
%        \addplot[color=purple,mark=square*,mark size=1.1,thick] table{results/logl_ray_0.txt};
%        \addplot[color=cyan,mark=*,mark size=1.1,thick] table{results/logl_ray_5.txt};
%        \addplot[color=orange,mark=triangle*,mark size=1.5,thick] table{results/logl_ray_10.txt};
%        \end{axis}
%        \end{tikzpicture}
%    }\hspace{1cm}
%    \subfloat[Gompertz distribution]{
%        \begin{tikzpicture}[trim axis left, trim axis right]
%        \begin{axis}
%        [
%        tiny,
%        width=0.4\columnwidth,
%        height=4.5cm,
%        legend pos=south east,
%        legend style={font=\scriptsize,nodes={scale=0.75, transform shape}},
%        grid,
%        y tick label style={
%            /pgf/number format/.cd,
%            fixed,
%            fixed zerofill,
%            precision=1,
%            /tikz/.cd
%        },
%        xlabel=$ N $,
%        %xticklabel style={rotate=90},
%        ylabel=$\log\mathcal{L}$,
%        ylabel shift = -4 pt,
%        %        ymax=0.22,
%        %        ymin=0.01,
%        xmin=0,
%        xmax=1000,
%        xtick={100,300,...,900},
%        restrict x to domain=0:900,
%        %        ytick={0.04,0.07,...,0.21},
%        legend entries={0\% censoring, 5\% censoring, 10\% censoring},
%        ]
%        \addplot[color=purple,mark=square*,mark size=1.1,thick] table{results/logl_gom_0.txt};
%        \addplot[color=cyan,mark=*,mark size=1.1,thick] table{results/logl_gom_5.txt};
%        \addplot[color=orange,mark=triangle*,mark size=1.5,thick] table{results/logl_gom_10.txt};
%        \end{axis}
%        \end{tikzpicture}
%    }
%    \caption{\npglm's achieved average log-likelihood on the separate test data vs the number of training samples ($N$) with different censoring ratios.}
%    \label{fig:syn-logl-n}
%\end{figure}

\begin{figure}[t]
    \subfloat[Rayleigh distribution]{
        \begin{tikzpicture}[trim axis left, trim axis right]
        \begin{axis}
        [
        tiny,
        width=0.4\columnwidth,
        height=4.5cm,
        legend pos=north east,
        legend style={font=\scriptsize,nodes={scale=0.75, transform shape}},
        grid,
        y tick label style={
            /pgf/number format/.cd,
            fixed,
            fixed zerofill,
            precision=2,
            /tikz/.cd
        },
        xlabel=$ N $,
        ylabel=MAE,
        ylabel shift = -4 pt,
        %xticklabel style={rotate=90},
        ymax=0.35,
        xmin=0,
        xmax=1000,
        ytick={0.05,0.10,...,0.35},
        xtick={100,300,...,900},
        restrict x to domain=0:900,
        legend entries={0\% censoring, 25\% censoring, 50\% censoring},
        ]
        \addplot[color=purple,mark=square*,mark size=1.1,thick] table{results/mae_ray.txt};
        \addplot[color=cyan,mark=*,mark size=1.1,thick] table{results/mae_ray_25.txt};
        \addplot[color=orange,mark=triangle*,mark size=1.5,thick] table{results/mae_ray_50.txt};
        \end{axis}
        \end{tikzpicture}
    }\hfil
    \subfloat[Gompertz distribution]{
        \begin{tikzpicture}[trim axis left, trim axis right]
        \begin{axis}
        [
        tiny,
        width=0.4\columnwidth,
        height=4.5cm,
        legend pos=north east,
        legend style={font=\scriptsize,nodes={scale=0.75, transform shape}},
        grid,
        y tick label style={
            /pgf/number format/.cd,
            fixed,
            fixed zerofill,
            precision=2,
            /tikz/.cd
        },
        xlabel=$ N $,
        %xticklabel style={rotate=90},
        ylabel=MAE,
        ylabel shift = -4 pt,
        ymax=0.22,
        ymin=0.01,
        xmin=0,
        xmax=1000,
        xtick={100,300,...,900},
        restrict x to domain=0:900,
        ytick={0.04,0.07,...,0.21},
        legend entries={0\% censoring, 25\% censoring, 50\% censoring},
        ]
        \addplot[color=purple,mark=square*,mark size=1.1,thick] table{results/mae_gom.txt};
        \addplot[color=cyan,mark=*,mark size=1.1,thick] table{results/mae_gom_25.txt};
        \addplot[color=orange,mark=triangle*,mark size=1.5,thick] table{results/mae_gom_50.txt};
        \end{axis}
        \end{tikzpicture}
    }
    \caption{\npglm's mean absolute error (MAE) vs the number of training samples ($N$) for different censoring ratios.}
    \label{fig:syn-mae-n}
\end{figure}

\begin{figure}[t]
    \subfloat[Rayleigh distribution]{
        \begin{tikzpicture}[trim axis left, trim axis right]
        \begin{axis}
        [
        tiny,
        width=0.4\columnwidth,
        height=4.5cm,
        legend pos=north east,
        legend style={font=\scriptsize,nodes={scale=0.75, transform shape}},
        grid,
        y tick label style={
            /pgf/number format/.cd,
            fixed,
            fixed zerofill,
            precision=2,
            /tikz/.cd
        },
        xlabel=$N_c$,
        %xticklabel style={rotate=90},
        ylabel=MAE,
        ylabel shift = -4 pt,
        ymax=0.2,
        ymin=0.06,
        %xmin=0,
        %xmax=2100,
        ytick={0.08,0.10,...,0.2},
        xtick={0,40,...,200},
        legend entries={$N_o=200$, $N_o=300$, $N_o=400$},
        ]
        \addplot[color=cyan,mark=*,mark size=1.1,thick] table{results/mae_ray_200.txt};
        \addplot[color=orange,mark=triangle*,mark size=1.5,thick] table{results/mae_ray_300.txt};
        \addplot[color=purple,mark=square*,mark size=1.1,thick] table{results/mae_ray_400.txt};
        \end{axis}
        \end{tikzpicture}
    }\hfil    
    \subfloat[Gompertz distribution]{
        \begin{tikzpicture}[trim axis left, trim axis right]
        \begin{axis}
        [
        tiny,
        width=0.4\columnwidth,
        height=4.5cm,
        legend pos=north east,
        legend style={font=\scriptsize,nodes={scale=0.75, transform shape}},
        grid,
        y tick label style={
            /pgf/number format/.cd,
            fixed,
            fixed zerofill,
            precision=2,
            /tikz/.cd
        },
        xlabel=$N_c$,
        %xticklabel style={rotate=90},
        ylabel=MAE,
        ylabel shift = -4 pt,
        ymax=0.24,
        ymin=0.03,
        %xmin=0,
        %xmax=2100,
        ytick={0.06,0.09,...,0.21},
        xtick={0,40,...,200},
        legend entries={$N_o=200$, $N_o=300$, $N_o=400$},
        ]
        \addplot[color=cyan,mark=*,mark size=1.1,thick] table{results/mae_gom_200.txt};
        \addplot[color=orange,mark=triangle*,mark size=1.5,thick] table{results/mae_gom_300.txt};
        \addplot[color=purple,mark=square*,mark size=1.1,thick] table{results/mae_gom_400.txt};
        \end{axis}
        \end{tikzpicture}
    }
    \caption{\npglm's mean absolute error (MAE) vs the number of censored samples ($N_c$) for different number of observed samples ($N_o$).}
    \label{fig:syn-mae-c}
\end{figure}





\begin{figure}[t]
    \subfloat[Rayleigh distribution]{
        \begin{tikzpicture}[trim axis left, trim axis right]
        \begin{axis}
        [
        tiny,
        width=0.4\columnwidth,
        height=4.5cm,
        legend pos=south east,
        legend style={font=\scriptsize,nodes={scale=0.75, transform shape}},
        grid,
%        y tick label style={
%            /pgf/number format/.cd,
%            fixed,
%            fixed zerofill,
%            precision=2,
%            /tikz/.cd
%        },
        xlabel=$N$,
        ylabel=$T\ \ (seconds)$,
        ylabel shift = -4 pt,
        %xticklabel style={rotate=90},
%        ymax=0.35,
%        xmin=3,
%        xmax=9,
%        ytick={0.05,0.10,...,0.35},
%        xtick={1000,10000,...,1000000000},
        xmode=log,
        ymode=log,
%        restrict x to domain=0:900,
        legend entries={0\% censoring, 25\% censoring, 50\% censoring},
        ]
        \addplot[color=purple,mark=square*,mark size=1.1,thick] table{results/time_ray_5.txt};
        \addplot[color=cyan,mark=*,mark size=1.1,thick] table{results/time_ray_25.txt};
        \addplot[color=orange,mark=triangle*,mark size=1.5,thick] table{results/time_ray_50.txt};
        \end{axis}
        \end{tikzpicture}
    }\hfil
    \subfloat[Gompertz distribution]{
        \begin{tikzpicture}[trim axis left, trim axis right]
        \begin{axis}
        [
        tiny,
        width=0.4\columnwidth,
        height=4.5cm,
        legend pos=south east,
        legend style={font=\scriptsize,nodes={scale=0.75, transform shape}},
        grid,
%        y tick label style={
%            /pgf/number format/.cd,
%            fixed,
%            fixed zerofill,
%            precision=2,
%            /tikz/.cd
%        },
        xlabel=$N$,
        ylabel=$T\ \ (seconds)$,
        ylabel shift = -4 pt,
        %xticklabel style={rotate=90},
        %        ymax=0.35,
        %        xmin=3,
        %        xmax=9,
        %        ytick={0.05,0.10,...,0.35},
        %        xtick={1000,10000,...,1000000000},
        xmode=log,
        ymode=log,
        legend entries={0\% censoring, 25\% censoring, 50\% censoring},
        ]
        \addplot[color=purple,mark=square*,mark size=1.1,thick] table{results/time_gom_5.txt};
        \addplot[color=cyan,mark=*,mark size=1.1,thick] table{results/time_gom_25.txt};
        \addplot[color=orange,mark=triangle*,mark size=1.5,thick] table{results/time_gom_50.txt};
        \end{axis}
        \end{tikzpicture}
    }
    \caption{\npglm's average running time ($T$) measured in seconds vs the number of training samples ($N$) in $\log-\log$ scale for different censoring ratios.}
    \label{fig:syn-time-n}
\end{figure}

\section{Experiments}\label{sec:results}

We apply \npglm with the proposed feature set on a number of real-world datasets to evaluate its effectiveness and compare its performance vis-\`a-vis state-of-the-art models. 

\subsection{Datasets}
%\subsubsection{DBLP}
%We use DBLP network from \cite{tang2008aminer}, which has both attributes of dynamicity and heterogeneity. The network contains four types of objects: authors, papers, venues, and terms. The network schema of this dataset is depicted in Fig~\ref{fig:schema:dblp}. Each paper is associated with a publication date, with a granularity of one year. Based on the publication venue of the papers, we limited the original DBLP dataset to those papers that are published in venues relative to the theoretical computer science. This resulted in having about 16k authors and 37k papers published from 1969 to 2016 in 38 venues. 

%The demographic statistics of both networks is presented in Table~\ref{table:dataset}.

\subsubsection{Delicious}
The first dataset we use in our experiments is the Delicious bookmarking dataset from \cite{Cantador:RecSys2011}, with a network schema presented in Fig~\ref{fig:schema:delicious}. It contains three types of objects, namely users, bookmarks, and tags, whose numbers are about 1.7k, 31k, and 22k, respectively. The dataset includes bookmarking timestamps from May 2006 to October 2010.

\subsubsection{MovieLens}
The second dataset has been extracted from MovieLens personalized movie recommendation website by \cite{harper2015}. The dataset comprises seven types of objects, that are users, movies, tags, genres, actors, directors, and countries, as illustrated by the network schema in Fig~\ref{fig:schema:movielens}. It contains about 1.4k users and 5.6k movies, with user-movie rating timestamps ranging from September 1997 to January 2009.

%\begin{table}[t]
%	\centering
%	\caption{Demographic Statistics of Real-World Datasets}
%	\label{table:dataset}
%	\scriptsize
%	\begin{tabu} to \columnwidth {l l X[l] X[l] X[r]}
%		\toprule
%		Dataset & Time Span & Entity & Title & Count\\
%		\midrule % In-table horizontal line
%		\multirow{8}{*}{DBLP} & \multirow{8}{2cm}{From 1969 to 2016}
%		& \multirow{4}{*}{Nodes}
%		& $Author$ & 15,929 \\ % Content row 1
%		& & & $Paper$ & 37,077  \\ % Content row 2
%		& & & $Venue$ & 38 \\ % Content row 3
%		& & & $Term$ & 12,028 \\ % Content row 3
%		\cmidrule{3-5}
%		& & \multirow{4}{*}{Links}
%		& write & 100,797 \\ % Content row 1
%		& & & cite & 165,904 \\ % Content row 2
%		& & & publish & 42,872 \\ % Content row 3
%		& & & mention & 284,156 \\ % Content row 3
%		
%		\midrule % In-table horizontal line
%		\multirow{6}{*}{Delicious} & \multirow{6}{2cm}{From May 2006 to Oct 2010}
%		& \multirow{3}{*}{Nodes}
%		& $User$ & 1,714 \\ % Content row 1
%		& & & $Tag$ & 21,956  \\ % Content row 2
%		& & & $Bookmark$ & 30,998 \\ % Content row 3
%		\cmidrule{3-5}
%		& & \multirow{3}{*}{Links}
%		& contact & 15,329 \\ % Content row 1
%		& & & post & 437,594 \\ % Content row 2
%		& & & has-tag & 437,594 \\ % Content row 3
%		
%		\midrule % In-table horizontal line
%		\multirow{13}{*}{MovieLens} & \multirow{13}{2cm}{From Sep 1997 to Jan 2009}
%		& \multirow{7}{*}{Nodes}
%		& $User$ & 1,421 \\ % Content row 1
%		& & & $Movie$ & 5,660  \\ % Content row 2
%		& & & $Actor$ & 6,176 \\ % Content row 3
%		& & & $Director$ & 2,401 \\ % Content row 3
%		& & & $Genre$ & 19 \\ % Content row 3
%		& & & $Tag$ & 5,561 \\ % Content row 3
%		& & & $Country$ & 63 \\ % Content row 3
%		\cmidrule{3-5}
%		& & \multirow{6}{*}{Links}
%		& rate & 855,599 \\ % Content row 1
%		& & & play-in & 231,743 \\ % Content row 2
%		& & & direct & 10,156 \\ % Content row 3
%		& & & has-genre & 20,810 \\ % Content row 3
%		& & & has-tag & 47,958 \\ % Content row 3
%		& & & produced-in & 10,198 \\ % Content row 3
%		\bottomrule % Bottom horizontal line
%	\end{tabu}
%\end{table}

\subsection{Experiment Settings}
\subsubsection{Comparison Methods}
To challenge the performance of \npglm, we use state-of-the-art GLM-based framework proposed in \cite{sun2012will} with Exponential, Rayleigh, and Weibull as distributions with different shapes, denoted as \textsc{Exp-Glm}, \textsc{Ray-Glm}, and \textsc{Wbl-Glm}, respectively. To examine the effect of considering the dynamicity of the network on performance of the models, we evaluate each one with two different feature sets: \emph{dynamic} and \emph{static}. Dynamic feature set is extracted using our proposed feature extraction framework, whereas static feature set only uses the very last snapshot of the network just before the beginning of the observation window. For all models, we consider the median of the distribution $f_T(t\mid\mb{x}_{test})$ as the predicted time for any test sample and then compare it to the ground truth time $t_{test}$.
\subsubsection{Performance Measures}
We measure the prediction error of different methods using various risk metrics, including Mean Absolute Error (MAE), Mean Relative Error (MRE), Root Mean Squared Error (RMSE), Mean Squared Logarithmic Error (MSLE), and Median Absolute Error (MDAE). We Also calculate the prediction accuracy using two metrics: (1) Maximum Threshold Prediction Accuracy (ACC), which measures for what fraction of samples, a model has a lower absolute error than a given threshold; and (2) Concordance Index (CI), which is the fraction of all pairs of samples whose predicted times are correctly ordered among all samples that can be ordered, and is considered as the generalization of the Area Under Receiver Operating Characteristic Curve (AUC) when we are dealing with censored data \cite{steck2008ranking}.
%\begin{itemize}
%\item Mean Absolute Error (MAE): This metric measures the expected absolute error between the predicted time values and the ground truth:
%\[MAE(\mb{t},\hat{\mb{t}}) = \frac{1}{N}\sum_{i=1}^{N}\left|t_i-\hat{t}_i\right|\]
%\item Mean Relative Error (MRE): This metric calculates the expected relative absolute error between the predicted time values and the ground truth:
%\[MRE(\mb{t},\hat{\mb{t}}) = \frac{1}{N}\sum_{i=1}^{N}\left|\frac{t_i-\hat{t}_i}{t_i}\right|\]
%\item Root Mean Squared Error (RMSE): This metric computes the root of the expected squared error between the predicted time values and the ground truth:
%\[RMSE(\mb{t},\hat{\mb{t}}) = \sqrt{\frac{1}{N}\sum_{i=1}^{N}\left(t_i-\hat{t}_i\right)^2}\]
%\item Mean Squared Logarithmic Error (MSLE): This measures the expected value of the squared logarithmic error between the predicted time values and the ground truth:
%\[RMSE(\mb{t},\hat{\mb{t}}) = \frac{1}{N}\sum_{i=1}^{N}\left(\log{(1+t_i)}-log{(1+\hat{t}_i)}\right)^2\]
%\item Median Absolute Error (MDAE): It is the median of the absolute errors between the predicted time values and the ground truth:
%\[MDAE(\mb{t},\hat{\mb{t}}) = median(\left|t_1-\hat{t}_1\right|\dots\left|t_N-\hat{t}_N\right|)\]
%\item Maximum Threshold Prediction Accuracy (ACC): This measures for what fraction of samples, a model have a lower absolute error than a given threshold:
%\[ACC(\mb{t},\hat{\mb{t}})=\frac{1}{N}\sum_{i=1}^{N}\mb{1}\left(\left|t_i-\hat{t}_i\right| < threshold\right)\]
%\item Concordance Index (CI): This metric is one of the most widely used performance measures for survival models that estimates how good a model performs at ranking predicted times \cite{harrell1982evaluating}. It can be seen as the fraction of all pairs of samples whose predicted times are correctly ordered among all samples that can be ordered, and is considered as the generalization of the Area Under Receiver Operating Characteristic Curve (AUC) when we are dealing with censored data \cite{steck2008ranking}.
%\end{itemize}

%We use 5-fold cross-validation and report the average results for all the experiments in this section. 
\subsubsection{Experiment Setup}
%For DBLP dataset, we confine the data samples to those authors who have published more than 5 papers in the feature extraction window of each experiment. Based on the author citation relation ($A\rightarrow P\rightarrow P\leftarrow A$) as the target relation, and using the similarity meta-paths in Table~\ref{table:meta}, we start the feature extraction process with 19 meta-paths.
For Delicious dataset, we aim to predict user-user relation ($U\leftrightarrow U$), which results in having 6 meta-paths for feature extraction, based on similarity meta-paths in Table~\ref{table:meta}.
For the MovieLens dataset, our goal is to predict user rate movie ($U\rightarrow M$) links, for which, we design 11 final meta-paths. We limit the actor list to the top three for each movie. To imply a notion of ``like'' relation between user and movie, we only consider ratings above 4 in scale of 5. For the sake of convenience, we convert the scale of time differences from timestamp to month in both datasets.

We implemented the LSTM autoencoder using Keras deep learning library. We used mean squared error loss function and Adadelta optimizer with default parameters. For all datasets, we set the dimension of the encoded feature as twice as the input dimension. For Np-Glm the data samples were ordered by their corresponding time variables, as the model needs the samples sorted by their recorded time. In all experiments, we pick an equal number of censored samples as the observed ones, uniformly at random. We use 5-fold cross-validation and report the average results.

\subsection{Experiment Results}
%In the rest of this section, we first assess how well different methods perform on various datasets and compare their performance based on different measures. Next, we analyze the effect of different parameters and problem configurations on the performance of competitive methods.

\begin{table}[t]
	\scriptsize
	\centering
	\caption{Performance Comparison of Different Methods on Different Datasets}
	\label{table:results}
	%\tiny
	\begin{tabu} to \columnwidth {c c l X[r] X[r] X[r] X[r] X[r] X[r]}
		\toprule
		%\cmidrule(l){2-3} \cmidrule{5-7}
		Dataset & Feature &
		Model &  MAE &   MRE &   RMSE &   MSLE &   MDAE &  CI \\
%		\midrule
%		\multirow{8}{*}{\rotatebox{90}{DBLP}}
%		& \multirow{4}{*}{\rotatebox{90}{Dynamic}}
%		& \npglm  &  $\bm{1.99}$ &  $\bm{0.95}$ &   $\bm{2.43}$ &   $\bm{0.30}$ &  $\bm{1.73}$ & $\bm{0.62}$ \\
%		& & \textsc{Wbl-Glm} &  2.33 &  1.10 &   2.85 &   0.36 &   2.08 & 0.58 \\
%		& & \textsc{Exp-Glm} &  3.11 &  1.39 &   3.88 &   0.52 &   2.58 & 0.50 \\
%		& & \textsc{Ray-Glm} &  4.02 &  1.83 &   4.70 &   0.66 &   3.72 & 0.35 \\
%		
%		\cmidrule{2-9}                                                                            
%		& \multirow{4}{*}{\rotatebox{90}{Static}}                                                  
%		& \npglm               &  2.76 &  1.35 &   3.07 &   0.44 &   2.88 & 0.26 \\
%		& & \textsc{Wbl-Glm}     &  2.81 &  1.38 &   3.16 &   0.45 &   2.88 & 0.48 \\
%		& & \textsc{Exp-Glm}     &  3.28 &  1.57 &   3.70 &   0.53 &   3.30 & 0.14 \\
%		& & \textsc{Ray-Glm}     &  5.04 &  2.28 &   5.26 &   0.85 &   5.12 & 0.01 \\
%		
		\midrule
		\multirow{8}{*}{\rotatebox{90}{Delicious}}
		& \multirow{4}{*}{\rotatebox{90}{Dynamic}}
		& \npglm  &  $\bm{2.10}$ &  $\bm{1.20}$ &   $\bm{2.55}$ &   $\bm{0.35}$ &   $\bm{2.05}$ & $\bm{0.70}$ \\
		& & \textsc{Wbl-Glm} &  2.37 &  1.31 &   2.89 &   0.40 &   2.16 & 0.57 \\
		& & \textsc{Exp-Glm} &  3.21 &  1.58 &   3.84 &   0.54 &   2.89 & 0.55 \\
		& & \textsc{Ray-Glm} &  3.90 &  2.07 &   4.66 &   0.68 &   3.91 & 0.40 \\
		
		\cmidrule{2-9}                 
		& \multirow{4}{*}{\rotatebox{90}{Static}}                                                  
		& \npglm               &  2.33 &  1.46 &   2.80 &   0.41 &   2.17 & 0.61 \\
		& & \textsc{Wbl-Glm}     &  2.65 &  1.62 &   3.23 &   0.47 &   2.26 & 0.43 \\
		& & \textsc{Exp-Glm}     &  3.35 &  1.91 &   4.17 &   0.59 &   2.75 & 0.35 \\
		& & \textsc{Ray-Glm}     &  4.81 &  2.61 &   5.27 &   0.85 &   4.28 & 0.12 \\
		
		\midrule
		\multirow{8}{*}{\rotatebox{90}{MovieLens}}
		& \multirow{4}{*}{\rotatebox{90}{Dynamic}}
		& \npglm  &  $\bm{2.48}$ &  $\bm{3.08}$ &   $\bm{3.04}$ &   $\bm{0.55}$ &  $\bm{2.14}$ & $\bm{0.70}$ \\
		& & \textsc{Wbl-Glm} &  3.06 &  3.61 &   3.79 &   0.65 &   2.60 & 0.56 \\
		& & \textsc{Exp-Glm} &  3.79 &  2.70 &   4.60 &   0.78 &   3.48 & 0.45 \\
		& & \textsc{Ray-Glm} &  4.98 &  3.58 &   5.63 &   1.05 &   4.83 & 0.33 \\
		
		\cmidrule{2-9}                                                                            
		& \multirow{4}{*}{\rotatebox{90}{Static}}                                                  
		& \npglm               &  2.92 &  3.44 &   3.45 &   0.67 &   3.36 & 0.50 \\
		& & \textsc{Wbl-Glm}     &  2.99 &  3.52 &   3.51 &   0.69 &   3.37 & 0.49 \\
		& & \textsc{Exp-Glm}     &  3.42 &  2.89 &   3.86 &   0.78 &   3.82 & 0.49 \\
		& & \textsc{Ray-Glm}     &  5.32 &  4.06 &   5.62 &   1.17 &   5.70 & 0.20 \\
		
		\bottomrule
	\end{tabu}
\end{table}

\begin{figure}[t]
	\centering
	%\hfill
%	\subfloat[DBLP]{
%		\begin{tikzpicture}[trim axis left, trim axis right]
%		\begin{axis}
%		[
%		tiny,
%		width=0.56\columnwidth,
%		height=4cm,
%		legend pos=north west,
%		legend style={font=\tiny,nodes={scale=0.75, transform shape}},
%		grid,
%		y tick label style={
%			/pgf/number format/.cd,
%			fixed,
%			fixed zerofill,
%			precision=1,
%			/tikz/.cd
%		},
%		xlabel=Absolute Error,
%		ylabel=Prediction Accuracy,
%		ylabel shift = -4 pt,
%		%xticklabel style={rotate=90},
%		ymax=1.1,
%		xmin=0,
%		xmax=3.5,
%		ytick={0.1,0.2,...,0.9,1.0},
%		xtick={0.5,1.0,...,3},
%		%		restrict x to domain=0:900,
%		legend entries={NP-GLM, WBL-GLM, EXP-GLM, RAY-GLM},
%		]
%		\addplot[color=purple,mark=square*,mark size=1.1,thick] table{results/db_np.txt};
%		\addplot[color=cyan,mark=triangle*,mark size=1.1,thick] table{results/db_wbl.txt};
%		\addplot[color=orange,mark=*,mark size=1.5,thick] table{results/db_exp.txt};
%		\addplot[color=green,mark=diamond*,mark size=1.5,thick] table{results/db_ray.txt};
%		\end{axis}
%		\end{tikzpicture}
%	}
%	\hfil
	\subfloat[Delicious]{
		\begin{tikzpicture}[trim axis left, trim axis right]
		\begin{axis}
		[
		tiny,
		width=0.5\columnwidth,
		height=3.8cm,
		legend pos=north west,
		legend style={font=\tiny,nodes={scale=0.75, transform shape}},
		grid,
		y tick label style={
			/pgf/number format/.cd,
			fixed,
			fixed zerofill,
			precision=1,
			/tikz/.cd
		},
		xlabel=Absolute Error,
		ylabel=Prediction Accuracy,
		ylabel shift = -4 pt,
		%xticklabel style={rotate=90},
		ymax=1.1,
		xmin=0,
		xmax=3.5,
		ytick={0.1,0.2,...,0.9,1.0},
		xtick={0.5,1.0,...,3},
		%		restrict x to domain=0:900,
		legend entries={NP-GLM, WBL-GLM, EXP-GLM, RAY-GLM},
		]
		\addplot[color=purple,mark=square*,mark size=1.1,thick] table{results/dl_np.txt};
		\addplot[color=cyan,mark=triangle*,mark size=1.1,thick] table{results/dl_wbl.txt};
		\addplot[color=orange,mark=*,mark size=1.5,thick] table{results/dl_exp.txt};
		\addplot[color=green,mark=diamond*,mark size=1.5,thick] table{results/dl_ray.txt};
		\end{axis}
		\end{tikzpicture}
	}
	\hfil
	\subfloat[MovieLens]{
		\begin{tikzpicture}[trim axis left, trim axis right]
		\begin{axis}
		[
		tiny,
		width=0.5\columnwidth,
		height=3.8cm,
		legend pos=north west,
		legend style={font=\tiny,nodes={scale=0.75, transform shape}},
		grid,
		y tick label style={
			/pgf/number format/.cd,
			fixed,
			fixed zerofill,
			precision=1,
			/tikz/.cd
		},
		xlabel=Absolute Error,
		ylabel=Prediction Accuracy,
		ylabel shift = -4 pt,
		%xticklabel style={rotate=90},
		ymax=1.1,
		xmin=0,
		xmax=3.5,
		ytick={0.1,0.2,...,0.9,1.0},
		xtick={0.5,1.0,...,3},
		%		restrict x to domain=0:900,
		legend entries={NP-GLM, WBL-GLM, EXP-GLM, RAY-GLM},
		]
		\addplot[color=purple,mark=square*,mark size=1.1,thick] table{results/mv_np.txt};
		\addplot[color=cyan,mark=triangle*,mark size=1.1,thick] table{results/mv_wbl.txt};
		\addplot[color=orange,mark=*,mark size=1.5,thick] table{results/mv_exp.txt};
		\addplot[color=green,mark=diamond*,mark size=1.5,thick] table{results/mv_ray.txt};
		\end{axis}
		\end{tikzpicture}
	}
	\caption{Prediction accuracy of different methods vs the maximum tolerated absolute error on different datasets.}
	\label{fig:real}
\end{figure}
\begin{figure}[t]
	\centering
%	\hfill
	\subfloat[Mean Absolute Error]{
		\begin{tikzpicture}[trim axis left, trim axis right]
		\begin{axis}
		[
		tiny,
		width=0.5\columnwidth,
		height=3.8cm,
		legend pos=north east,
		legend style={font=\tiny,nodes={scale=0.75, transform shape}},
		grid,
		y tick label style={
			/pgf/number format/.cd,
			fixed,
			fixed zerofill,
			precision=1,
			/tikz/.cd
		},
		xlabel=\# Snapshots,
		ylabel=MAE,
		ylabel shift = -4 pt,
		%xticklabel style={rotate=90},
		%		ymax=1.1,
		xmin=0,
		xmax=21,
		%		ytick={0.1,0.2,...,0.9,1.0},
		xtick={3,6,...,18},
		%		restrict x to domain=0:900,
		legend entries={NP-GLM, WBL-GLM},
		]
		\addplot[color=purple,mark=square*,mark size=1.1,thick] table{results/dl_snap_mae_np.txt};
		\addplot[color=cyan,mark=triangle*,mark size=1.1,thick] table{results/dl_snap_mae_wbl.txt};
		\end{axis}
		\end{tikzpicture}
	}
	\hfil
	\subfloat[Concordance Index]{
		\begin{tikzpicture}[trim axis left, trim axis right]
		\begin{axis}
		[
		tiny,
		width=0.5\columnwidth,
		height=3.8cm,
		legend pos=north west,
		legend style={font=\tiny,nodes={scale=0.75, transform shape}},
		grid,
		y tick label style={
			/pgf/number format/.cd,
			fixed,
			fixed zerofill,
			precision=1,
			/tikz/.cd
		},
		xlabel=\# Snapshots,
		ylabel=CI,
		ylabel shift = -4 pt,
		%xticklabel style={rotate=90},
		%		ymax=1.1,
		xmin=0,
		%		xmax=3.5,
		%		ytick={0.1,0.2,...,0.9,1.0},
		xtick={3,6,...,18},
		%		restrict x to domain=0:900,
		legend entries={NP-GLM, WBL-GLM},
		]
		\addplot[color=purple,mark=square*,mark size=1.1,thick] table{results/dl_snap_ci_np.txt};
		\addplot[color=cyan,mark=triangle*,mark size=1.1,thick] table{results/dl_snap_ci_wbl.txt};
		\end{axis}
		\end{tikzpicture}
	}
	\caption{Effect of choosing different number of snapshots on performance of different methods using Delicious dataset.}
	\label{fig:snaps:delicious}
\end{figure}
\begin{figure}[t]
	\centering
%	\hfill
	\subfloat[Mean Absolute Error]{
		\begin{tikzpicture}[trim axis left, trim axis right]
		\begin{axis}
		[
		tiny,
		width=0.5\columnwidth,
		height=3.8cm,
		legend pos=north east,
		legend style={font=\tiny,nodes={scale=0.75, transform shape}},
		grid,
		y tick label style={
			/pgf/number format/.cd,
			fixed,
			fixed zerofill,
			precision=1,
			/tikz/.cd
		},
		xlabel=\# Snapshots,
		ylabel=MAE,
		ylabel shift = -4 pt,
		%xticklabel style={rotate=90},
		%				ymax=18,ymin=10,
		xmin=0,
		%		xmax=3.5,
		%		ytick={0.1,0.2,...,0.9,1.0},
		xtick={3,6,...,18},
		%		restrict x to domain=0:900,
		legend entries={NP-GLM, WBL-GLM},
		]
		\addplot[color=purple,mark=square*,mark size=1.1,thick] table{results/mv_snap_mae_np.txt};
		\addplot[color=cyan,mark=triangle*,mark size=1.1,thick] table{results/mv_snap_mae_wbl.txt};
		\end{axis}
		\end{tikzpicture}
	}
	\hfil
	\subfloat[Concordance Index]{
		\begin{tikzpicture}[trim axis left, trim axis right]
		\begin{axis}
		[
		tiny,
		width=0.5\columnwidth,
		height=3.8cm,
		legend pos=north west,
		legend style={font=\tiny,nodes={scale=0.75, transform shape}},
		grid,
		y tick label style={
			/pgf/number format/.cd,
			fixed,
			fixed zerofill,
			precision=1,
			/tikz/.cd
		},
		xlabel=\# Snapshots,
		ylabel=CI,
		ylabel shift = -4 pt,
		%xticklabel style={rotate=90},
		ymax=0.9,ymin=0.2,
		xmin=0,
		%		xmax=3.5,
		%		ytick={0.1,0.2,...,0.9,1.0},
		xtick={3,6,...,18},
		%		restrict x to domain=0:900,
		legend entries={NP-GLM, WBL-GLM},
		]
		\addplot[color=purple,mark=square*,mark size=1.1,thick] table{results/mv_snap_ci_np.txt};
		\addplot[color=cyan,mark=triangle*,mark size=1.1,thick] table{results/mv_snap_ci_wbl.txt};
		\end{axis}
		\end{tikzpicture}
	}
	\caption{Effect of choosing different number of snapshots on performance of different methods using MovieLens dataset.}
	\label{fig:snaps:MovieLens}
\end{figure}
\begin{figure}[t]
	\centering
%	\hfill
	\subfloat[Mean Absolute Error]{
		\begin{tikzpicture}[trim axis left, trim axis right]
		\begin{axis}
		[
		tiny,
		width=0.5\columnwidth,
		height=3.8cm,
		legend pos=north east,
		legend style={font=\tiny,nodes={scale=0.75, transform shape}},
		grid,
		y tick label style={
			/pgf/number format/.cd,
			fixed,
			fixed zerofill,
			precision=1,
			/tikz/.cd
		},
		xlabel=$\Delta$,
		ylabel=MAE,
		ylabel shift = -4 pt,
		%xticklabel style={rotate=90},
		%		ymax=1.1,
		xmin=0,
		xmax=3.5,
		%		ytick={0.1,0.2,...,0.9,1.0},
		xtick={0.5,1.0,...,3},
		%		restrict x to domain=0:900,
		legend entries={NP-GLM, WBL-GLM},
		]
		\addplot[color=purple,mark=square*,mark size=1.1,thick] table{results/delta_mae_np.txt};
		\addplot[color=cyan,mark=triangle*,mark size=1.1,thick] table{results/delta_mae_wbl.txt};
		\end{axis}
		\end{tikzpicture}
	}
	\hfil
	\subfloat[Concordance Index]{
		\begin{tikzpicture}[trim axis left, trim axis right]
		\begin{axis}
		[
		tiny,
		width=0.5\columnwidth,
		height=3.8cm,
		legend pos=north west,
		legend style={font=\tiny,nodes={scale=0.75, transform shape}},
		grid,
		y tick label style={
			/pgf/number format/.cd,
			fixed,
			fixed zerofill,
			precision=1,
			/tikz/.cd
		},
		xlabel=$\Delta$,
		ylabel=CI,
		ylabel shift = -4 pt,
		%xticklabel style={rotate=90},
		%		ymax=1.1,
		xmin=0,
		xmax=3.5,
		%		ytick={0.1,0.2,...,0.9,1.0},
		xtick={0.5,1.0,...,3},
		%		restrict x to domain=0:900,
		legend entries={NP-GLM, WBL-GLM},
		]
		\addplot[color=purple,mark=square*,mark size=1.1,thick] table{results/delta_ci_np.txt};
		\addplot[color=cyan,mark=triangle*,mark size=1.1,thick] table{results/delta_ci_wbl.txt};
		\end{axis}
		\end{tikzpicture}
	}
	\caption{Effect of choosing different values for $\Delta$ on performance of different methods using Delicious dataset.}
	\label{fig:delta:delicious}
\end{figure}
\begin{figure}[t]
	\centering
%	\hfill
	\subfloat[Mean Absolute Error]{
		\begin{tikzpicture}[trim axis left, trim axis right]
		\begin{axis}
		[
		tiny,
		width=0.5\columnwidth,
		height=3.8cm,
		legend pos=north east,
		legend style={font=\tiny,nodes={scale=0.75, transform shape}},
		grid,
		y tick label style={
			/pgf/number format/.cd,
			fixed,
			fixed zerofill,
			precision=1,
			/tikz/.cd
		},
		xlabel=$\Delta$,
		ylabel=MAE,
		ylabel shift = -4 pt,
		%xticklabel style={rotate=90},
		ymax=18,ymin=10,
		xmin=0,
		xmax=3.5,
		%		ytick={0.1,0.2,...,0.9,1.0},
		xtick={0.5,1.0,...,3},
		%		restrict x to domain=0:900,
		legend entries={NP-GLM, WBL-GLM},
		]
		\addplot[color=purple,mark=square*,mark size=1.1,thick] table{results/mv_delta_mae_np.txt};
		\addplot[color=cyan,mark=triangle*,mark size=1.1,thick] table{results/mv_delta_mae_wbl.txt};
		\end{axis}
		\end{tikzpicture}
	}
	\hfil
	\subfloat[Concordance Index]{
		\begin{tikzpicture}[trim axis left, trim axis right]
		\begin{axis}
		[
		tiny,
		width=0.5\columnwidth,
		height=3.8cm,
		legend pos=north west,
		legend style={font=\tiny,nodes={scale=0.75, transform shape}},
		grid,
		y tick label style={
			/pgf/number format/.cd,
			fixed,
			fixed zerofill,
			precision=1,
			/tikz/.cd
		},
		xlabel=$\Delta$,
		ylabel=CI,
		ylabel shift = -4 pt,
		%xticklabel style={rotate=90},
		ymax=0.9,ymin=0.2,
		xmin=0,
		xmax=3.5,
		%		ytick={0.1,0.2,...,0.9,1.0},
		xtick={0.5,1.0,...,3},
		%		restrict x to domain=0:900,
		legend entries={NP-GLM, WBL-GLM},
		]
		\addplot[color=purple,mark=square*,mark size=1.1,thick] table{results/mv_delta_ci_np.txt};
		\addplot[color=cyan,mark=triangle*,mark size=1.1,thick] table{results/mv_delta_ci_wbl.txt};
		\end{axis}
		\end{tikzpicture}
	}
	\caption{Effect of choosing different values for $\Delta$ on performance of different methods using MovieLens dataset.}
	\label{fig:delta:MovieLens}
\end{figure}

\subsubsection{Comparative Performance Analysis}
In the first set of experiments, we evaluate the prediction power of different models on Delicious and MovieLens datasets. For feature extraction in all cases, we set $\Delta=1$, $\Omega=6$, and $k=12$. MAE, MRE, RMSE, MSLE, MDAE and CI of all models using both dynamic and static feature sets have been shown in Table~\ref{table:results}. We see that in both datasets, \npglm using the dynamic features is superior to the other models under all performance measures. For instance, our model \npglm can obtain an MAE of 2.10 for Delicious dataset, which is 11\% lower than the MAE obtained by its closest competitor, \textsc{Wbl-Glm}. As of CI, \npglm achieves 0.70 on Delicious, which is 23\% better than \textsc{Wbl-Glm}. 
On MovieLens dataset, \npglm improves MAE and CI by 19\% and 25\%, respectively, relative to \textsc{Wbl-Glm}. Comparable results hold for other measures as well. Moreover, in this table it is evident that using the dynamic features has a positive impact on the performance of all models.

In the next experiment, we investigate the performance of different methods using the dynamic feature set under maximum threshold prediction accuracy. In other words, to evaluate the prediction accuracy of a model, we record the fraction of test samples for which the difference between their true times and predicted ones are lower than a given threshold, called \emph{tolerated error}. The parameter settings ($\Delta$, $\Omega$, $k$) is the same as the previous experiment. The results for different tolerated errors in range $\{0.5, 1.0, \dots, 3.0\}$ were plotted in Fig~\ref{fig:real}, which demonstrate that \npglm can achieve a higher accuracy in all cases relative to other baselines.


\subsubsection{Parameter Setting Analysis}
The performance of different models is influenced by two parameters, the number of snapshots $k$, and the time difference between snapshots $\Delta$, as these parameters determine the length of the feature extraction window. In this set of experiments, we investigate how these parameters affect the performance of our model \npglm and its competitor \textsc{Wbl-Glm} on Delicious and MovieLens datasets. 

The effect of increasing the number of snapshots on achieved MAE and CI by \npglm and \textsc{Wbl-Glm} over Delicious and MovieLens datasets is illustrated in Fig~\ref{fig:snaps:delicious} and Fig~\ref{fig:snaps:MovieLens}, respectively. For both datasets, we set $\Delta=1.5$ and $\Omega=18$ and varied the number of snapshots in the range of 3 to 18. As we can see in both figures, increasing the number of snapshots results in lower prediction error and higher accuracy. This is due to the fact that as the number of snapshots grows, a longer history of the network is taken into account. 

Finally, the impact of choosing different values for $\Delta$ is analyzed on the performance of \npglm and \textsc{Wbl-Glm} in terms of MAE and CI. The results for Delicious and MovieLens datasets are depicted in Fig~\ref{fig:delta:delicious} and Fig~\ref{fig:delta:MovieLens}, respectively. In this experiment, number of snapshots and observation window length are accordingly set to 6 and 24. Different values of $\Delta$ are selected from the set $\{0.5,1.0,\dots,3.0\}$. As illustrated in both figures, by increasing  $\Delta$ up to an extent, we witness that the performance of models improves gradually. That is because increasing the value of $\Delta$ leads to a wider feature extraction window. However, since the number of snapshots is constant, we see no performance improvement when the value of $\Delta$ becomes greater than a certain threshold. This is due to the fact that short term temporal evolution of the network will be ignored when the value of $\Delta$ becomes too wide.
 




\section{Related Works}\label{sec:related}
\newcommand{\etal}{\textit{et~al}.}

The problem of link prediction has been studied extensively in recent years and many approaches have been proposed to solve this problem \cite{wang2015link}.
Previous work on time-aware link prediction have mostly considered temporality in analyzing the long-term network trend over time \cite{dhote2013survey}. Authors in \cite{potgieter2009temporality} have shown that temporal metrics are extremely valuable new contribution to link prediction, and should be used in future applications. 
%\cite{tylenda2009towards} incorporated temporal information available on evolving social networks for link prediction tasks and proposed a novel node-centric approach to the evaluation of link prediction. 
Dunlavy \etal{} focused on the problem of periodic temporal link prediction \cite{dunlavy2011temporal}. They concentrated on bipartite graphs that evolve over time and also considered weighted matrix that contained multilayer data and tensor-based methods for predicting future links.
Oyama \etal{} solved the problem of cross-temporal link prediction, in which the links among nodes in different time frames are inferred \cite{oyama2011cross}.
%{\"O}zcan \etal{} proposed a novel link prediction method for evolving networks based on NARX neural network \cite{ozcan2016temporal}. They take the correlation between the quasi-local similarity measures and temporal evolutions of link occurrences information into account by using NARX for multivariate time series forecasting.
Yu \etal{} developed a novel temporal matrix factorization model to explicitly represent the network as a function of time \cite{yu2017temporally}. They provided results for link prediction as an specific example and showed that their model performs better than the state-of-the-art techniques.

The most relevant works to this study are available in \cite{sun2012will, aggarwal2012dynamic, sett2017temporal}. Aggarwal \etal{} \cite{aggarwal2012dynamic} tackle the link prediction problem in both dynamic and heterogeneous information networks using a dynamic clustering approach alongside with content-based and structural models. However, they aim to solve the conventional link prediction problem, not the continuous-time link prediction studied in this paper. In \cite{sett2017temporal}, the authors proposed a feature set, called TMLP, well suited for link prediction in dynamic and heterogeneous information networks. Although their proposed feature set cope with both dynamicity and heterogeneity of the network, again it cannot be generalized for continuous-time link prediction and is only designed for solving the simpler link prediction problem.

Most of the aforementioned works answered the question of \emph{whether} a link will appear in the network. To the best of our knowledge, the only work that has focused on the continuous-time relationship prediction problem is proposed by Sun \etal{} \cite{sun2012will}, in which a generalized linear model based framework is suggested to model the relationship building time. 
%They consider the building time of links as independent random variables coming from a pre-specified distribution and model the expectation as a function of a linear predictor of the extracted topological features. 
A shortcoming of this model is that we need to exactly specify the underlying distribution of relationship building times. We came over this problem by learning the distribution from the data using a non-parametric solution. Furthermore, we considered the temporal dynamics of the network which has been entirely ignored in their work.


\section{Conclusion}\label{sec:conclusion}
In this paper, we studied the problem of continuous-time relationship prediction in both dynamic and heterogeneous information networks. To effectively tackle this problem, we first introduced a novel feature extraction framework based on meta-path modeling and recurrent neural network autoencoders to systematically extract features that take both the temporal dynamics and heterogeneous characteristics of the network into account for solving the continuous-time relationship problem. We then proposed a supervised non-parametric model, called \npglm, which exploits the extracted features to predict the relationship building time in information networks. The strength of our model is that it does not impose any significant assumptions on the underlying distribution of the relationship building time given its features, but tries to infer it from the data via a non-parametric approach. Extensive experiments conducted on a synthetic dataset and real-world datasets from DBLP, Delicious, and MovieLens demonstrated the correctness of our method and its effectiveness in predicting the relationship building time.

{\color{red}For future work, we would like to design a unified architecture to combine feature extraction step with the learning algorithm in an integrated deep learning framework. Moreover, although the propsed method is able to scale to large information networks with thousands of nodes, it is not currently extensible to web-scale information networks where the number of nodes is in the scale of hundreds of millions. Learning temporal non-parametric models within an extremely huge dataset is a challenging problem and is an interesting and important future work. As calculating meta-path-based features are the primary computational bottleneck of our method, to make the learning process scalable, we set to investigate node embedding and approximation techniques.}


\bibliographystyle{ACM-Reference-Format}
\bibliography{references}

\end{document}
