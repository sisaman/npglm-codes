\section{Feature Extraction Framework}\label{sec:features}

In this section, we present our framework to extract features which is designed to have three major characteristics: First, it effectively considers different type of nodes and links available in a heterogeneous information network and regard their impact on the building time of the target relationship. Second, it takes the temporal dynamics of the network into account and leverage the network evolution history instead of simply aggregating it into a single snapshot. Finally, the extracted features are suitable for not only the link prediction problem, but also the generalized \emph{relationship prediction}. We will incorporate these features in the proposed non-parametric model in Section~\ref{sec:method} to solve the continuous-time relationship prediction problem.

\subsection{Data Preparation For Feature Extraction}
To solve the problem of continuous-time relationship prediction in dynamic networks, we need to pay attention to the temporal history of the network data from two different point of views. First, we have to mind the evolution history of the network for the feature extraction, so that the extracted features reflect the changes made in the network over time. Second, we have to specify the exact relationship building time for each pair of nodes, because our goal is to propose a supervised method to predict a continuous variable, which in this case is the relationship building time. Hence, for each sample pair of nodes, we need a feature vector $\mb{x}$, associated with a target variable $t$ which indicates the building time of the target relationship between them.

Suppose that we have a dynamic network $G^{t_0,t_1}$ which has been recorded in the interval $(t_0, t_1]$. According to Fig~\ref{fig:timeline}, we split this interval into two parts: the first part for extracting the feature $\mb{x}$, and the second for determining the target variable $t$. We refer to the first interval as \emph{Feature Extraction Window} whose length is denoted by $\Phi$, and the second as \emph{Observation Window} whose length is denoted by $\Omega$. Now, based on the existence state of target relationships in Observation Window, we categorize them into the following three different groups:

\begin{enumerate}
	\item Relationships that are already formed before beginning of Observation Window (formed in the Feature Extraction Window).
	\item Relationships that will be built in the Observation Window for the first time (not existing before).
	\item Relationships that will not form at all (neither in Feature Extraction Window nor Observation Window).
\end{enumerate}

Those pair of nodes that act as the starting and ending nodes of the relationships in the 2nd and 3rd categories constitute our data samples, and will be used in the learning procedure. For such pairs, we extract their feature vector $\mb{x}$ using the history available in Feature Extraction Window. For each node pair in the 2nd category, we have seen that the target relationship between them is created at a time like $t_r\in(t_0+\Phi,t_1]$. So we set $t=t_r-(t_0+\Phi)$ as the time it takes for the relationship to form since the beginning of the Observation Window. For these samples, we also set an auxiliary variable $y=1$ which indicates that we have \emph{observed} their exact building time. On the other hand, For node pairs in the 3rd category, we haven't seen their exact building time, but we know the it is definitely after $t_1$. For such samples, which we call them \emph{censored} ones, we set $t=t_1-(t_0+\Phi)$ which is equal to the length of the Observation Window $\Omega$, and set $y=0$ to indicate that the true relationship building time is in fact beyond the recorded time. These type of samples though look useless, are also of interest because their features will give us some information about their time falling after $t_1$. As a result, each final sample is associated with a triple $(\mb{x},y,t)$ representing its feature vector, observation status, and the time it takes for the target relationship to form, respectively.

%In Section \ref{sec:method}, we propose \npglm which is a supervised method to relate $x_l$ to $t_l$ by estimating $f_T(t_l\mid x_l)$ in a non-parametric fashion.

%Here is an toy example in a bibliographic network: Suppose that we have the data of the papers published between the years 1990 and 2010. For all papers, we have their authors, venue (where they are published), indexing terms, and their references. For each author pair, the goal is to predict by when one of them will cite another, if she has not done yet. To this end, we pick an intermediary year such as 2000 as pivot, and split the the data into two part. The paper that are published just before the year 2000 will belong to the Feature Extraction Window, and the rest of the papers will fall within Observation Window. Now, for each pair of authors who did not cite each other in Feature Extraction Window which 

\begin{table}[t]
	\centering
	\caption{Similarity Meta-Paths Used for Feature Extraction \cite{sun2012will}}
	\label{table:meta}
	\scriptsize
	\begin{tabu} to \columnwidth {X[c] X[l]}
		\toprule
		Meta-Path & Semantic Meaning \\
		\midrule
		$A\rightarrow~P\leftarrow~A$ & Authors co-write a paper\\
		$A\rightarrow~P\rightarrow~A\leftarrow~P\leftarrow~A$ & Authors have common co-authors\\
		$A\rightarrow~P\rightarrow~V\leftarrow~P\leftarrow~A$ & Authors publish in the same venues\\
		$A\rightarrow~P\rightarrow~T\leftarrow~P\leftarrow~A$ & Authors use the same terms\\
		$A\rightarrow~P\rightarrow~P\leftarrow~P\leftarrow~A$ & Authors cite the same papers\\
		$A\rightarrow~P\leftarrow~P\rightarrow~P\leftarrow~A$ & Authors are cited by the same papers\\
		\bottomrule
	\end{tabu}
\end{table}

\subsection{Meta-Path-based Dynamic Feature Extraction}
In this part, we describe how to utilize the temporal history of the network in the Feature Extraction Window in order to extract features for continuous-time relationship prediction problem. We extend the feature set proposed in \cite{sun2012will} for heterogeneous information networks, by including the capability to exploit the temporal dynamics of the network as well. Hereby, we begin by defining the concept of meta-path \cite{sun2011pathsim}:

\begin{definition}[Meta-Path]
In a heterogeneous information network, a meta-path is a directed path following the graph of the network schema to describe the general relations that can be derived from the network. Formally speaking, given a network schema $\mc{S}_G=(\mc{V}, \mc{E})$, the sequence $\nu_1\xrightarrow{\varepsilon_1}\nu_2\xrightarrow{\varepsilon_2}\dots\nu_{k-1}\xrightarrow{\varepsilon_{k-1}}\nu_k$ is a meta-path defined on $S_G$ where $\nu_i\in \mc{V}$ and $\varepsilon_i\in \mc{E}$.
\end{definition} 

Meta-paths are commonly used in heterogeneous information networks to describe multi-typed relations which have concrete semantic meanings. For example, in the bibliographic network whose schema were show in Fig~\ref{fig:schema}, we can define the co-authorship relation with the following meta-path:
\[Author\xrightarrow{write}Paper\xleftarrow{write}Author\]
or simply by $A\rightarrow P\leftarrow A$. Also, the author citation relation which is used as the target relation in this paper, can be specified as:
\[Author\xrightarrow{write}Paper\xrightarrow{cite}Paper\xleftarrow{write}Author\]
or abbreviated as $A\rightarrow P\rightarrow P\leftarrow A$.

Among the possible meta-paths that can be defined on a network schema, there are some that capture the similarity between two nodes. For example, $A\rightarrow P\leftarrow A$ which is the co-authorship relation between two authors in a bibliographic network, creates a sense of similarity between two \emph{Author} nodes. These type of meta-paths, called \emph{similarity meta-paths}, are widely used to define topological features for link prediction problem in heterogeneous networks \cite{sun2011co, zhang2014meta, 7752228}. Some examples of similarity meta-paths that capture the topological similarity between two \emph{Author} nodes in the bibliographic network, are available in Table~\ref{table:meta}.

The concept of similarity meta-paths can be extended to define topological features suitable for relationship prediction problem, where we have a target relation. Here we follow the same approach as in \cite{sun2012will} which suggests the following three meta-path-based blocks to describe features for relationship prediction problem, given a target relation between two nodes of type $A$ and $B$:
\begin{enumerate}
\small
\item $A\xrsquigarrow{similarity}A\xrsquigarrow{target}B$
\item $A\xrsquigarrow{target}B\xrsquigarrow{similarity}A$
\item $A\xrsquigarrow{relation}C\xrsquigarrow{relation}B$
\end{enumerate}
where $\rightsquigarrow$ denotes a meta-path, with labels \emph{similarity} and \emph{target} denoting a similarity meta-path and the target relationship, respectively. The \emph{relation} label denotes an arbitrary meta-path relating two nodes of possibly different types. The first block tells that there are some nodes of type $A$ similar to a single node of the same type that has made the target relationship with a node of type $B$. Therefore, those similar nodes may also form the target relation with the type $B$ node. An analogous intuition is behind the second block. For the third, it says that some nodes of type $A$ are in relation with some type $C$ nodes, which are themselves in relation with some nodes of type $B$. Hence, it is likely that type $A$ nodes form some relationships, such as the target relationship, with type $B$ nodes.

For our case study with the bibliographic network, for the target relation we use $A\rightarrow P\rightarrow P\leftarrow A$ as meta-path denoting the author citation relation. In Addition, Paper-cite-Author ($P\rightarrow P\rightarrow A$) and Author-cite-Paper ($A\rightarrow P\rightarrow P$) are also used as the arbitrary relations, and the similarity meta-paths of Table~\ref{table:meta} to define the features for author citation relationship prediction.

After specifying the suitable meta-paths, we need a method to quantify them as features. Here, due to the dynamicity of the network, different links are emerging and vanishing from the network over time. Therefore, the quantifying method must handle this dynamicity. Here, we formally define \emph{Time-Aware Meta-Path-based Features}:

\begin{definition}[Time-Aware Meta-Path-based Feature]
Suppose that we are given a dynamic heterogeneous network $G^{t_0,t_1}$ along with its network schema $\mc{S}_G=(\mc{V}, \mc{E})$, and a target Relation $A\rightsquigarrow B$. For a given pair of nodes $a\in A$ and $b\in B$, and a meta-path $\Psi=\nu_1\xrightarrow{\varepsilon_1}\nu_2\xrightarrow{\varepsilon_2}\dots\nu_{k-1}\xrightarrow{\varepsilon_{k-1}}\nu_k$ defined on $\mc{S}_G$, the time-aware meta-path-based feature in the time interval $(t_p, t_q)$ is calculated as:
\begin{equation}
\begin{split}
&f_{a,b}^\Psi(t_p,t_q)=I(a,A)I(b,B)\\
&\sum_{n_1\in\nu_1,n_2\in\nu_2,\dots,n_k\in\nu_k}\prod_{i=1}^{k-1}I\Big((n_i,n_{i+1}),\varepsilon_i\Big)I\left((n_i,n_{i+1}),t_p,t_q\right)
\end{split}
\end{equation}
where:
\[
I(u,\mc{U})=
\begin{cases}
1&\quad\text{if } u\in \mc{U}\\
0&\quad\text{otherwise}\\
\end{cases}
\]
checks whether a network entity (node or link) $u$ belongs to type $\mc{U}$, and:
\[
I(e,t_p,t_q)=
\begin{cases}
1&\quad\text{if } t_p< t_b(e)\le t_q<t_d(e)\\
0&\quad\text{otherwise}\\
\end{cases}
\]
checks whether a link exists in the interval $(t_p,t_q)$.
\end{definition}

\begin{figure}
	\definecolor{blue}{HTML}{84CECC}
	\definecolor{darkblue}{HTML}{375D81}
	\definecolor{green}{HTML}{3F7F47}
	\begin{chronology}[align=left, startyear=0,stopyear=200, width=\columnwidth, height=1pt, startdate=false, stopdate=false, arrowwidth=4pt, arrowheight=3pt]
		\scriptsize
		\chronoevent[date=false]{10}{$t_0$}
		\chronoevent[date=false]{40}{$t_0+\Delta$}
		\chronoevent[date=false]{70}{$t_0+2\Delta$}
		\chronoevent[date=false,mark=false]{100}{$\dots$}
		\chronoevent[date=false]{130}{$t_0+k\Delta$}
		\chronoevent[date=false]{190}{$t_1$}
		\chronoperiode[color=darkblue, startdate=false, bottomdepth=2pt, topheight=5pt, textdepth=8pt, stopdate=false]{10}{40}{$\Delta$}
		\chronoperiode[color=blue, startdate=false, bottomdepth=10pt, topheight=15pt, textdepth=-15pt, stopdate=false]{10}{129}{Feature Extraction Window $(\Phi=k\Delta)$}
		\chronoperiode[color=green, startdate=false, bottomdepth=10pt, topheight=15pt, textdepth=-15pt, stopdate=false]{131}{190}{Observation Window $(\Omega)$}
	\end{chronology}
	\caption{The evolutionary timeline of the network data.}
	\label{fig:timeline}
\end{figure}

By using the above definition, we will be able to quantify the number of instances of a particular meta-path in a specific time period. If we set this time period equal to the Feature Extraction Window, it is as though we are aggregating the network into a single snapshot observed at time $t_0+\Phi$. In order to avoid such an aggregation, we divide the Feature Extraction Window into a sequence of $k$ contiguous intervals of a constant size $\Delta$, as shown in Fig.~\ref{fig:timeline}. By doing this discretization, we intend to extract time-aware features in each sub-window, and then weigh them in a way that decay exponentially with the time, giving the recent features a higher weight. With this in mind, we define \emph{Dynamic Meta-Path-based Features} as follows:

\begin{definition}[Dynamic Meta-Path-based Feature]
Suppose that we are given a dynamic heterogeneous network $G^{t_0,t_1}$ with a Feature Extraction Window of size $\Phi$, along with its network schema $\mc{S}_G=(\mc{V}, \mc{E})$ and a target Relation $A\rightsquigarrow B$. Also suppose that the Feature Extraction Window is divided into $k$ fragments of size $\Delta$. For a given pair of nodes $a\in A$ and $b\in B$ in $G^{t_0,t_0+\Phi}$, and a meta-path $\Psi$ defined on $\mc{S}_G$, the dynamic meta-path-based feature of $(a,b)$ is calculated as:
\begin{equation}
x_{a,b}=\sum_{i=0}^{k-1}f_{a,b}^\Psi\Big(t_0+i\Delta, t_0+(i+1)\Delta\Big)\exp\{-\mu_\Psi i\Delta\}
\end{equation}
where $\mu_\Psi\ge0$ is the rate parameter for the meta-path $\Psi$ which controls the decay of $f_{a,b}^\Psi$ over time.
\end{definition}

With the rate parameter $\mu_\Psi$, we can adjust the decay rate of time-aware features over time. As $\mu_\Psi\rightarrow 0$, the dynamic meta-path-based feature will give the same weight to all $\Delta$-sized windows so we would again end up in the case of a single static snapshot observed at the end of the Feature Extraction Window. On the contrary, as $\mu_\Psi\rightarrow\infty$ only the very recent $\Delta$-sized window will take part in the feature extraction and the rest of them will be ignored. With different rate parameters for different meta-paths we can exactly tune the weight of each feature over time.


