\section{Introduction}\label{sec:intro}
Link prediction is the problem of prognosticating a certain relationship, like interaction or collaboration, between two entities in a networked system that are not connected already \cite{lu2011link}. Due to the popularity and ubiquity of networked systems in real world, such as social, economical, or biological networks, this problem has attracted a considerable attention in recent years and has found its applications in various interdisciplinary domains, such as viral marketing, bioinformatics, recommender systems, and social network analysis \cite{wasserman1994social}. For example, suggesting new friends in an online social network \cite{liben2007link} or predicting drug-target interactions in a biological network \cite{chen2012drug} are two quite different problems, but can both cast as the prediction task of friendship links and drug-target links respectively.

The problem of link prediction has a long literature and is studied extensively in the last decade. Initial works on link prediction problem mostly concentrated on homogeneous networks, which are composed of single type of nodes connected by links of the same type \cite{liben2007link, wang2007local, lichtenwalter2010new}. However, many of the today's networks, such as online social networks or bibliographic networks, are inherently \emph{heterogeneous}, in which multiple types of nodes are interconnected using multiple types of links \cite{taskar2004link, shi2017survey}. For example, a bibliographic network may contain author, paper, venue, etc. as different node types; and write, publish, cite, and so on as diverse link types that bind nodes of different types to each other. In these heterogeneous networks, the concept of a link can be generalized to a relationship, which can be constructed by combining different links with different types. For instance, author-cite-paper relationship can be defined in a bibliographic network as a combination of author-write-paper and paper-cite-paper links. Analogously, one can generalize the link prediction to \emph{relationship prediction} in heterogeneous networks which tries to predict complex relationships instead of links \cite{sun2012will}.

While most of the studies on the link/relationship prediction in heterogeneous networks utilize a static snapshot of the underlying network, many of these networks are \emph{dynamic} in nature, which means that new nodes and linkages are continually added to the network, and some existing nodes and links may be removed from the network over time. For example, in online social networks such as Facebook, new users are joining in the network every day, and new friendship links are being added to the network, gradually. This dynamic characteristic causes the structure of the network to change and evolve over time, and taking these changes into account can significantly boost the quality of link prediction task \cite{potgieter2009temporality}.

In recent years, newer studies have shifted from traditional link prediction on static and homogeneous networks toward newer domains, considering heterogeneity and dynamicity of the networks \cite{dong2012link, davis2011multi, 7752228, hajibagheri2016leveraging, moradabadi2017novel}. However, most of these works merely focus on one of these aspects, disregarding the other. Although there are quite a few studies that address both the challenges of heterogeneity and dynamicity \cite{aggarwal2012dynamic, sett2017temporal}, to the best of our knowledge, all of them have ultimately formulated the link prediction problem as a binary classification task, i.e. predicting \emph{whether} a link will appear in the network in the future. However, in dynamic networks, new links are continually appearing over time. So a much more interesting problem, which we call it \emph{continuous-time link prediction} in this paper, is to predict \emph{when} a link will emerge or appear between two nodes in the network. Examples of this problem include predicting the time at which two individuals become friends in a social network, or the time when two authors collaborate on writing a paper in a bibliographic network \cite{sun2012will}. Inferring the link formation time in advance can be very useful in many concrete applications. For example, if a social network recommender system could predict the relationship building time between two people, then it can issue a friendship suggestion close to that time since it will have a relatively higher chance to be accepted. Good continuous-time link prediction results will lead to denser connections among users, and can greatly improve users' engagement that is the ultimate goal of online social networks \cite{kwak2010twitter}.

In this paper, we aim to solve the problem of continuous-time relationship prediction, in which we forecast the relationship building time between two nodes in a dynamic and heterogeneous environment. This problem is very challenging from the technical perspective, and cannot be solved trivially for three main reasons. First, the formulation of continuous-time relationship prediction is quite different from conventional link prediction due to the involvement of temporal dynamics of the network and the necessity of considering network evolution time-line. Second, we only know the building time of those relationships that are already present at the network and for the rest of them that are yet to happen, which are excessive in number versus the existing ones, we lack such information. Finally, as opposed to the works concerning the binary link prediction, there are very rare works in the literature on continuous-time link prediction that attempt to answer the ``when'' question. As far as we know, the only work that has studied the continuous-time relationship prediction problem so far, is proposed by Sun et al. \cite{sun2012will}. They infer a probability distribution over time for each pair of nodes given their features, and answer time-related queries about the relationship building time between the two nodes using the inferred distribution. However, the drawback of their method, not to mention neglecting the temporal dynamics of the network, is that it mainly relies on the assumption that relationship building times are coming from a certain probability distribution which must be fixed beforehand. This assumption though simplifying is very restrictive because, in real applications, this distribution is unknown and considering any specific one as a priori could be far from reality or limit the solution generality.

In order to address the above challenges, we propose a supervised non-parametric method to solve the problem of continuous-time relationship prediction. To this end, we first formally define the continuous-time relationship prediction problem and formulate the approach to solve it generally. Then, we introduce our meta-path-based feature extraction framework which leverages both heterogeneity and dynamicity of information networks, well-suited for relationship prediction task. Next, we present \emph{Non-Parametric Generalized Linear Model} (\npglm) which models the distribution of relationship building time given the features extracted before. The strength of this non-parametric model is that it is capable of learning the underlying distribution of the relationship building time, as well as the amount of contribution of each extracted feature in the network. Inferring such probability distribution, we propose an inference algorithm to answer queries, like the most probable time by which a relationship will appear between two nodes, or the probability of relationship creation between them during a specific period. Finally, we conduct comprehensive experiments over a synthetic dataset to verify the correctness of \npglm's learning algorithm, and on a real-world DBLP bibliographic network dataset to demonstrate the effectiveness of the proposed method in predicting the relationship building time versus the relevant baselines. As a summary, we can enumerate our major contributions as follows:

\begin{itemize}
\item The proposed meta-path-based feature extraction framework can utilize heterogeneity of the data as well as capturing the temporal dynamics of the network.
\item Our non-parametric model takes a unique approach toward learning the underlying distribution of relationship building time without imposing any significant assumption on the problem.
\item Extensive evaluations over both synthetic data and DBLP real-world bibliographic network is performed to indicate the effectiveness of the proposed method. 
\item To the best of our knowledge, this paper is the first one which studies the continuous-time relationship prediction problem in both dynamic and heterogeneous network configuration.
\end{itemize}

The rest of this paper is organized as follows. In Section \ref{sec:problem}, we provide introductory backgrounds on the concept and formally define the problem of continuous-time relationship prediction. Then in Section \ref{sec:features}, we introduce our meta-path-based feature extraction framework. Next, we go through the details of our proposed \npglm method in Section \ref{sec:method}, explaining its learning method and how it answers inference queries. Experimental results are described in Section \ref{sec:results}. Section \ref{sec:related} discusses the related works and finally in Section \ref{sec:conclusion}, we conclude the paper.
